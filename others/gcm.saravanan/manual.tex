%**start of header
% TeX preface file
 
%set magnification
 
\magnification=\magstephalf
 
%set page size and position
 
%\hsize=6.5 true in
%\vsize=8.9 true in
%\hoffset=0.0 true in
%\voffset=0.0 true in
 
%set line and paragraph spacing
 
\openup 1\jot
\parskip=6pt plus 3pt
 
% General TeX macros
 
\def\Title#1{ \vskip 0.5in \centerline{\bf #1} \vskip 0.25in }
                                            % title of document
\def\Author#1{ \bigskip \centerline{\sl #1} \bigskip }
                                            % author's name
\def\Address#1\period{ {\it \tabskip=0pt plus 1fil
  \halign to \hsize{\hfil##\hfil\cr#1} \bigskip } }
                                            % address (end lines with \cr)
\def\Section#1.#2.
     {\par\bigbreak {\bf\noindent #1. #2} \par\nobreak \bigskip}
                                            % section heading
\def\Subsection#1.#2.
     {\par\bigbreak {\it\noindent #1. #2} \par\nobreak \bigskip}
                                            % sub-section heading
\def\Appendix#1.#2.
     {\par\bigbreak {\bf\noindent Appendix #1. #2} \par\nobreak \bigskip}
                                            % appendix heading
 
\def\References{\par\bigbreak \centerline{\bf References} \par\nobreak
                \bigskip}
                                           % references heading
\def\pref#1/#2/#3.#4,#5,#6.
     {\par\hang\noindent #1, #2:#3.{\it #4,}{\bf #5},#6. \smallskip}
                                            % paper reference
\def\bref#1/#2/#3.#4.
     {\par\hang\noindent #1, #2:{\it #3.}#4. \smallskip}
                                            % book reference
\def\oref#1\cr{\par\hang\noindent #1 \smallskip}
                                            % other reference
 
\def\litem{\par\hang\noindent}              % long item
 
\def\nullcs#1{{}}                           % null control sequence
 
\newcount\eqcount                           % initialize equation counter
\eqcount=0
\def\neweq{\global\advance\eqcount by 1 \number\eqcount}
                                            % obtain new eq. number
\def\cureq{\number\eqcount}                 % current equation number
\def\releq#1{\count255 = \eqcount \advance\count255 by #1 \number\count255}
                                            % relative equation number
 
\def\v#1{{\bf#1}}                           % vector boldface
\def\td[#1,#2]{{d#1\over{d#2}}}             % total derivative
\def\pd[#1,#2]{{\partial#1\over{\partial#2}}}
                                            % partial derivative
 
% Standard journal abbreviations
 
\def\JAS{J.\ Atmos.\ Sci.}
\def\QJRMS{Quart.\ J.\ Roy.\ Meteor.\ Soc.}
\def\MWR{Mon.\ Wea.\ Rev.}
\def\JMSJ{J.\ Meteor.\ Soc.\ Japan}
 
% Macros related to primitive equations
 
\def\half{{1 \over 2}}                      % one-half
\def\Dp{{\Delta p}}                         % \Delta p
\def\Dt{{\Delta t}}                         % \Delta t
 
\def\Dcap{{\widehat D}}                     % \widehat D
\def\Phicap{{\widehat \Phi}}                % \widehat \Phi
\def\zetacap{{\widehat \zeta}}              % \widehat \zeta
\def\Thetabar{{\overline \Theta}}           % \overline \Theta
\def\Thetacap{{\widehat \Theta}}            % \widehat \Theta
 
\def\Htilde{{\tilde H}}                     % \tilde H
\def\Phitilde{{\tilde \Phi}}                % \tilde \Phi
 
\def\uvec{{\v u}}                           % vector u
\def\kvec{{\v k}}                           % vertical unit vector
 
\def\Tvec{{\vec \Theta}}                    % \Theta vector
\def\Dvec{{\vec D}}                         % D vector
\def\Wvec{{\vec \omega}}                    % \omega vector
\def\Hvec{{ {\vec H}_{T^R} }}               % H_{T^R} vector
\def\xivec{{\vec \xi}}                      % \xi vector
\def\Dcapvec{{\vec \Dcap}}                  % \Dcap vector
\def\Phicapvec{{\vec \Phicap}}              % \Phicap vector
\def\Xvec{{\vec X}}                         % X vector
 
\def\Htildevec{{ {\vec \Htilde}_{T^R} }}    % \Htilde vector
\def\Phitildevec{{\vec \Phitilde}}          % \Phitilde vector
\def\Hcorrvec{{ {\vec H}_{corr} }}          % H correction vector
 
\def\Mx[#1,#2]{{ M_{#1 \rightarrow #2} }}   % mapping matrix
 
\def\vdiv#1{{ {\left(\pd[#1,p]\right)}_l }} % vertical divergence
\def\tt[#1,#2]{{ {\left(\pd[#1,t]\right)}^{#2} }}
                                            % time-tendency
\def\delh{\mathop{{\v \nabla}_H}}           % horizontal gradient operator
\def\deleight{\mathop{{\v \nabla}_H^8}}     % scale-selective damping operator
\def\div{\mathop{\rm div}}                  % horizontal divergence
\def\curlz{\mathop{\rm curl_z}}             % vertical component of curl
\def\delsq{\mathop{\nabla_H^2}}             % horizontal Laplacian
 
\def\Pmn[#1,#2]{ P_{#1,#2} }                % associated Legendre polynomial
\def\Ymn[#1,#2]{ Y_{#1,#2} }                % spherical harmonic
%**end of header
 
\Title{A Mechanistic Spectral Primitive Equation Model using Pressure
       Coordinates}
 
\Author{R.~Saravanan}
\Address Department of Applied Mathematics and Theoretical Physics\cr
         University of Cambridge\cr
         Silver Street, Cambridge CB3 9EW, U.K.\cr
         E-mail: svn@atm.amtp.cam.ac.uk\cr\period
 
\bigskip
 
This document and the software described in it are distributed free for
research purposes. The software comes with absolutely no warranties
whatsoever. This software may be freely copied, and may also be modified,
provided the modified software continues to be freely available under the
same terms. This software may not be used for any commercial purpose.
 
\vfill\eject
 
\item{1.} Introduction
\item{2.} Primitive equations in pressure coordinates
\item{3.} Vertical discretization
\itemitem{a.} Vertical grid
\itemitem{b.} Boundary conditions
\itemitem{c.} Conservation properties
\itemitem{d.} Linearization about reference vertical stratification
\itemitem{e.} Vorticity/divergence form
\item{4.} Time integration
\itemitem{a.} Semi-implicit leap-frog scheme
\itemitem{b.} Time-filtering
\item{5.} Horizontal discretization
\item{6.} Miscellaneous details
\itemitem{a.} Passive tracers
\itemitem{b.} Choice of reference stratification
\itemitem{c.} Forcing/damping terms
\item{7.} Fortran implementation
\itemitem{a.} Overview
\itemitem{b.} Module {\it splib}
\itemitem{c.} Module {\it prognos}
\itemitem{d.} Compilation and customization
\itemitem{e.} Initialization and time-marching
 
\medskip Appendices \medskip
 
\item{A.} Symbols and notation
\item{B.} Fortran to Symbols ``dictionary''
\item{C.} Modifications to allow forcing at bottom boundary
 
\medskip References \bigskip
 
 
\Section 1. Introduction.
 
This document describes the implementation of a very simple 3-dimensional
primitive equation model, using pressure coordinates as described by Lorenz
(1960). The advantage of using pressure coordinates is that the equations of
motion take on a particularly simple form. The main disadvantage is that we
cannot easily apply a realistic boundary condition at the lower boundary,
i.e., it is not really possible to include the effects of surface topography.
The model we are about to describe assumes that the planetary surface is
essentially smooth, and that there is sufficient drag acting near the lower
boundary to make the details of the lower boundary condition quite
unimportant. This model would not be suitable for ``realistic'' simulation of
the Earth's climate (especially in the troposphere), but could be quite useful
for mechanistic models of isolated phenomena, and for studying idealized
planetary atmospheres.
 
The model is very flexible in terms of the number of pressure levels, level
thicknesses, and zonal versus meridional resolution. The number of levels can
be any integer $\ge 2$. The levels can have any combination of thicknesses.
The horizontal discretization uses spectral truncation, which is usually
triangular. For simplicity, the spectral transforms are carried out all the
latitudes simultaneously. Any order of triangular truncation is allowed, but
very high orders may be impractical in terms of memory requirements, because
of the way the spectral transforms are carried out. The range of zonal
wavenumbers may be further restricted within the range of triangular
truncation, allowing anisotropic integrations with only very few zonal
wavenumbers being resolved. The extreme case of axially symmetric horizontal
representation can also be handled, although not too efficiently. The model
can also handle an arbitrary number of passive tracers, using simple spectral
advection.
 
The model incorporates virtually no physical parameterizations of any kind.
But simple forms of scale-selective damping, and Newtonian/Rayleigh friction
are provided to facilitate mechanistic simulations.
 
One of the unusual features of this model is that the vertically integrated
divergence over the whole atmosphere is constrained to be identically zero. In
other words, divergent shallow water modes with barotropic vertical structure
are excluded. This leads to increased computational stability of the model.
This also means that the model needs virtually no other boundary conditions,
and only interior forcing/damping parameters need to be specified.  But this
constraint can easily be relaxed to allow ``geopotential forcing'' at the
lower boundary, as would be necessary for models of the middle atmosphere
alone. (The necessary modifications are described in Appendix C, although they
have not been implemented as yet.)
 
 
\Section 2. Primitive equations in pressure coordinates.
 
The primitive equations for a dry rotating spherical shallow compressible
hydrostatic atmosphere in Lagrangian form may be written in pressure
coordinates as (e.g. see Holton, 1979)
 
$$ \eqalignno{
\td[\uvec, t]  &= - f\kvec\times\uvec - \delh\Phi + {\v F}_u & (\neweq a) \cr
\td[\Theta, t] &= F_T                                        & (\cureq b) \cr
0 &= \delh\cdot\uvec + \pd[\omega, p]                        & (\cureq c) \cr
\pd[\Phi, \zeta]   &= - C_p\Theta                            & (\cureq d) \cr
}$$
 
where $\zeta = {(p/p_s)}^\kappa$ denotes an auxiliary vertical coordinate,
$\Theta$ denotes potential temperature, ${\v F_u}$ denotes mechanical
forcing/damping terms, and $F_T$ denotes thermal forcing/damping terms.
Otherwise the notation is fairly standard, and the reader is referred to
Appendix A for a complete definition of all symbols.
 
Defining the operators
$$ \curlz(\;) \equiv (\kvec\times\delh) \cdot (\;) ;\quad
   \div(\;)   \equiv \delh \cdot (\;)
$$
 
we define the relative vorticity $\xi$ and divergence $D$ as
 
$$ D   \equiv \div\uvec;\quad
   \xi \equiv \curlz\uvec
$$
 
Then we can write the primitive equations in Eulerian form as follows:
 
$$ \eqalignno{
\pd[\uvec, t]      &= - \delh\Phi - \delh({1\over2}\uvec^2)
- (f+\xi)\kvec\times\uvec - D\uvec - \pd[(\omega\uvec), p]   & (\neweq a) \cr
\pd[\Theta,t]      &= - \div(\Theta\uvec) -  \pd[(\omega\Theta), p]
                                                             & (\cureq b) \cr
\pd[\omega, p]     &= - D                                    & (\cureq c) \cr
\pd[\Phi, \zeta]   &= - C_p\Theta                            & (\cureq d) \cr
}$$
 
Note that we have omitted the forcing/damping terms. We shall continue to
ignore these terms in our discussion of spatial discretization. We will
re-introduce some of the damping terms later when we discuss the time-marching
scheme.
 
\Section 3. Vertical discretization.
 
\Subsection a. Vertical grid.
 
We assume that the domain of the model atmosphere extends upward from a
reference pressure level $p=p_s$ (``surface'') to the level $p=0$ (``top'' of
the atmosphere). Following Lorenz (1960), we divide the pressure-interval
$[0,p_s]$ into $L$ (possibly unequal) sub-intervals, each of extent $\Dp_l$.
 
$$p_s = \sum_l \Dp_l, \; l= 1, \ldots, L
$$
 
We may then think of various quantities as being defined in the middle of
these $L$ sub-intervals, which we refer to as {\it levels} (or {\it
full-levels}). The levels are numbered, starting from the
uppermost-level, as $1, \ldots, L$.  The pressure $p_l$ at the full-levels is
defined by
 
$$ p_l = {1\over2}\Dp_l + \sum_{l'=1}^{l-1} \Dp_{l'}
$$
 
It is then convenient to introduce the concept of {\it half-levels} which lie
at the boundaries of the L pressure sub-intervals. There would be $L-1$
half-levels which lie in between the $L$ full-levels. These we will refer to
as the {\it interior} half-levels. There would also be two more half-levels,
one at the top of the atmosphere, and another at the surface. These we will
refer to as the boundary half-levels. The half-levels are numbered, starting
from the top, as $\half, 1+\half, \ldots, L-\half, L+\half$.
 
$$ p_{l+\half} = \sum_{l'=1}^l \Dp_{l'};\qquad p_\half=0,\quad p_{L+\half}=p_s.
$$
 
We take the prognostic quantities to be the horizontal velocity $\uvec_l$ and
potential temperature $\Theta_l$ defined at each of the $L$ full-levels.  The
continuity equation (2c) then suggests that it would be natural to define
``pressure velocity'' $\omega_{l+\half}$ at the half-levels.  Next we
introduce the {\it bar} and {\it cap} operators:
 
$$ {\overline q}_{l+\half} \equiv { q_{l+1} + q_l \over 2}
$$
 
$$ {\widehat  q}_{l+\half} \equiv { q_{l+1} - q_l \over 2}
$$
where $q$ is some quantity defined at full-levels.
(Note that the above operators are not defined at the boundary half-levels)
 
We then define the vertical flux of prognostic quantities at the interior
half-levels as
 
$$ {\v V}_{u,l+\half} \equiv \omega_{l+\half} {\overline  \uvec}_{l+\half}
$$
 
$$      V_{T,l+\half} \equiv \omega_{l+\half} \Thetabar_{l+\half}
$$
 
For any vertical flux $V_{q,l+\half}$, we define the vertical divergence of
this flux at the full-levels as
 
$$ \vdiv{V_q} \equiv {V_{q,l+\half}-V_{q,l-\half}\over\Dp_l}
$$
 
We define the {\it vertical integral} of a quantity $q$ as being $\sum_{l=1}^L
\Dp_l \, q_l$, which is essentially a mass-weighted integral. In particular,
we note that
 
$$ \sum_{l=1}^L \Dp_l \vdiv{V_q} = V_{q,L+\half} - V_{q,\half}
$$
 
We assume that the geopotential $\Phi_l$ is also defined at the
full-levels. Defining $\zeta_l \equiv (p_l/p_s)^\kappa$, we choose to
discretise the hydrostatic equation (2d) as follows:
 
$$ {\Phi_{l+1} - \Phi_l \over \zeta_{l+1} - \zeta_l} \equiv
    - C_p \Thetabar_{l+\half}
$$
 
We can then write the vertically discretized version of (\cureq) as
 
$$ \eqalignno{
\pd[\uvec_l, t]   &= - \delh\Phi_l - \delh E_l - {\v H}_{u,l}
                                                             & (\neweq a) \cr
\pd[\Theta_l,t]   &= - H_{T,l}                               & (\cureq b) \cr
D_l &= - {\omega_{l+\half}-\omega_{l-\half}\over\Dp_l}       & (\cureq c) \cr
\Phicap_{l+\half} &=
                - C_p \zetacap_{l+\half} \Thetabar_{l+\half} & (\cureq d) \cr
\noalign{\hbox{where}}
{\v H}_{u,l} &\equiv  (f+\xi_l)\kvec\times\uvec_l + D_l\uvec_l
                    + \vdiv{{\v V}_u}                        & (\neweq a) \cr
H_{T,l}      &\equiv  \div(\Theta_l\uvec_l) + \vdiv{V_T}     & (\cureq b) \cr
E_l          &\equiv  {1 \over 2} \uvec^2_l                  & (\cureq c) \cr
}$$
 
The auxiliary quantity ${\v H}_u$ represents a part of the momentum flux
divergence, $H_T$ represents the heat flux divergence, and $E$ represents the
kinetic energy.
 
 
\Subsection b. Boundary conditions.
 
We choose to set the pressure-velocity $\omega$ to zero at the upper and lower
boundaries of the atmosphere:
 
$$\omega_\half = \omega_{L+\half} = 0
$$
 
This choice then leads to the constraint that the vertically integrated
divergence is zero. i.e.
 
$$ \sum_{l=1}^L \Dp_l\,D_l = \omega_\half - \omega_{L+\half} = 0
$$
 
This choice also allows us to set the vertical fluxes at the boundaries to
zero. i.e
 
$$ {\v V}_{u,\half} = {\v V}_{u,L+\half} = V_{T,\half} = V_{T,L+\half} = 0
$$
 
 
\Subsection c. Conservation properties.
 
It should be obvious from inspecting (\releq{-1}) that the vertical
discretization preserves the global conservation properties of absolute
angular momentum $(\Omega a \cos\phi + u)\cos\phi$ and potential temperature
$\Theta$.  Now we proceed to show that the global integral of total energy is
also conserved (in the absence of forcing/damping). Taking the dot-product of
\uvec\ with (\releq{-1}a), we obtain
 
$$\pd[E_l,t] = - \uvec_l \cdot \delh \Phi_l - \uvec_l \cdot \delh E_l
                    - D {\uvec_l}^2  - \uvec_l \cdot \vdiv{{\v V}_u}
$$
 
After some rearranging, this can be written as
 
$$\pd[E_l,t] = - \div (E_l \uvec_l) - \div (\Phi_l \uvec_l) - \vdiv{V_E}
               + \Phi_l D_l \eqno (\neweq)
$$
 
where $V_{E,l+\half} \equiv \omega_{l+\half} \half(\uvec_{l+1} \cdot \uvec_l) $
 
If we further define
$ V_{\Phi,l+\half} \equiv \omega_{l+\half} {\overline \Phi}_{l+\half} $
then we get
 
$$ \vdiv{V_\Phi}= - \Phi_l D_l + {  \omega_{l+\half}\Phicap_{l+\half}
                                  + \omega_{l-\half}\Phicap_{l-\half}
                                    \over \Dp_l }
$$
 
If we also define
$ V_{ {\bar\zeta}\,\Thetabar,l+\half} \equiv
  \omega_{l+\half} {\bar\zeta}_{l+\half} \Thetabar_{l+\half} $
then we get
 
$$ C_p \vdiv{V_{{\bar\zeta}\,\Thetabar}} = C_p \zeta_l \vdiv{V_T}
  - { \omega_{l+\half}\Phicap_{l+\half} + \omega_{l-\half}\Phicap_{l-\half}
      \over \Dp_l }
$$
 
This allows us to write the kinetic energy tendency equation (\cureq) as
 
$$\pd[E_l,t] = - \div \{(E_l + \Phi_l) \uvec_l\} - \vdiv{V_E} - \vdiv{V_\Phi}
               - C_p \vdiv{V_{{\bar\zeta}\,\Thetabar}} + C_p \zeta_l \vdiv{V_T}
               \eqno (\neweq)
$$
 
If we multiply the potential temperature tendency equation (\releq{-3}b) by
$C_p
\zeta$, we obtain
 
$$\pd[C_p\zeta_l\Theta_l,t] = - \div(C_p \zeta_l \Theta_l \uvec_l)
                 - C_p \zeta_l \vdiv{V_T} \eqno (\neweq)
$$
 
From (\releq{-1}) and (\cureq) it is clear that the vertical truncation
preserves the global conservation property of the total energy
$(E + C_p\zeta\Theta)$.
 
Another quantity conserved by this vertical truncation is $\Theta^2$. If we
multiply (\releq{-4}b) by $\Theta$, we obtain
 
$$ \pd[\half\Theta^2_l,t] = - \div(\half\Theta^2_l\uvec_l)
                            - \vdiv{V_{\half\Theta^2}} \eqno (\neweq)
$$
 
where $ V_{\half\Theta^2,l+\half} \equiv
       \omega_{l+\half} \half \Theta_{l+1} \Theta_l  $
 
 
\Subsection d. Linearization about reference vertical stratification.
 
For the purposes of the semi-implicit time-stepping scheme that we will be
using, it is necessary to express the $\Theta$ distribution in terms of
deviations from some reference vertical profile $\Theta^R(p)$. We express
$\Theta(\lambda,\phi,p,t)$ as
 
$$\Theta = \Theta^R(p) + \Theta'(\lambda,\phi,p,t)
$$
 
Then we also decompose the heat fluxes as follows:
 
$$\eqalignno{
H_T             &= H_{T^R} + H_{T'}\cr
H_{T',l}        &= \div(\Theta'_l\uvec_l) + \vdiv{V_{T'}}
                   \hphantom{----------}\cr
H_{T^R,l}       &= \Theta^R_l D_l + \vdiv{V_{T^R}}\cr
V_{T',l+\half}  &= \omega_{l+\half} {\Thetabar'}_{l+\half}\cr
}$$
 
We can re-express $H_{T^R}$ as
 
$$\eqalignno{
H_{T^R,l} &= {  \Thetacap^R_{l+\half} \omega_{l+\half}
                +\Thetacap^R_{l-\half} \omega_{l-\half} \over \Dp_l}\cr
          &= \sum_{l'=1}^{L-1} (\delta_{l',l} + \delta_{l',l-1})
              \Thetacap^R_{l'+\half} \omega_{l'+\half}\cr
}$$
 
(Here we have used the fact that $\omega_\half = \omega_{L+\half} = 0$.)
 
Defining a column vector of length $L-1$:
$\Wvec = (\omega_{1+\half},\ldots,\omega_{L-\half}) $,
and a column vector of length $L$:
$\Hvec = (H_{T^R,1},\ldots,H_{T^R,L}) $,
we can write the above equation in matrix form as
 
$$ \Hvec = \Mx[\omega,H] \> \Wvec
$$
 
where $\Mx[\omega,H]$ is an $L \times (L-1)$ matrix defined by
 
$${\left[ \Mx[\omega,H] \right]}_{l,l'} =
  \cases{ {\Thetacap^R_{l'+\half} \over \Dp_l} & if $l'=l$ or $l'=l-1$;\cr
          0                                    & otherwise.\cr}
$$
 
$$  \Mx[\omega,H] = \pmatrix{
\Thetacap^R_{1+\half}\over\Dp_1&0&0                              &\cdots&0&0\cr
\Thetacap^R_{1+\half}\over\Dp_2&\Thetacap^R_{2+\half}\over\Dp_2&0&\cdots&0&0\cr
0&\Thetacap^R_{2+\half}\over\Dp_3&\Thetacap^R_{3+\half}\over\Dp_3&\cdots&0&0\cr
\vdots&\vdots&\vdots&\ddots&\vdots&\vdots\cr
0&0&0&\cdots
&\Thetacap^R_{L-1-\half}\over\Dp_{L-1}&\Thetacap^R_{L-\half}\over\Dp_{L-1}\cr
0&0&0&\cdots&0                        &\Thetacap^R_{L-\half}\over\Dp_L\cr
}$$
 
Next we note that
 
$$ \omega_{l'+\half} = - D_{l'} \Dp_{l'} + \omega_{l'-\half}
                     = - \sum_{l''=1}^{l'} \Dp_{l''} D_{l''}
$$
 
Defining a column vector of length $L$:
$\Dvec = (D_1,\ldots,D_L) $, we can rewrite the above equation in matrix form
as follows:
 
$$ \Wvec = \Mx[D,\omega] \> \Dvec
$$
 
where $\Mx[D,\omega]$ is an $(L-1) \times L$ matrix defined by
 
$${\left[ \Mx[D,\omega] \right]}_{l',l''} =
  \cases{ -\Dp_{l''} & if $l''\le l'$;\cr
          0          & otherwise.\cr}
$$
 
$$  \Mx[D,\omega] = \pmatrix{
-\Dp_1&      0&      0& \cdots&          0&      0\cr
-\Dp_1& -\Dp_2&      0& \cdots&          0&      0\cr
-\Dp_1& -\Dp_2& -\Dp_3& \cdots&          0&      0\cr
\vdots& \vdots& \vdots& \ddots&     \vdots& \vdots\cr
-\Dp_1& -\Dp_2& -\Dp_3& \cdots& -\Dp_{L-1}&      0\cr
}$$
 
This allows us to define the compound matrix $\Mx[D,H] = \Mx[\omega,H] \>
\Mx[D,\omega]$.
 
\Subsection e. Vorticity/divergence form.
 
Rather than work with the discretised momentum equations, we prefer to work
with the time-tendency equations for vorticity and divergence. This is
motivated by the fact that when representing the horizontal variation of
quantities using spherical harmonics, it is much easier to deal with
scalars (such as vorticity) than with components of vectors. We apply the
$\div$ and $\curlz$ operators to (3a) to obtain the following prognostic
equations
 
$$ \eqalignno{
\pd[\xi_l, t]   &= - \curlz {\v H}_{u,l}                     & (\neweq a) \cr
\pd[D_l, t]     &= - \div   {\v H}_{u,l}
                   - \delsq E_l - \delsq \Phi_l              & (\cureq b) \cr
}$$
 
Since the quantities $D_l$ are not all independent, it is convenient to work
with the $L-1$ independent quantities $\Dcap_{l+\half}$ at the interior
half-levels. We can express $D$ in terms of $\Dcap$ as follows:
 
$$ D_l = 2 \Dcap_{l-\half} + D_{l-1}
       = \sum_{l'=1}^{l-1} 2 \Dcap_{l'+\half} + D_1
$$
 
Now we compute the vertically integrated divergence
 
$$ \sum_{l'=1}^L D_{l'}\Dp_{l'} = \sum_{l'=1}^{L-1}
   2 \Dcap_{l'+\half}(p_{L+\half} - p_{l'+\half}) + D_1 \, p_{L+\half}= 0
$$
 
This gives the following expression for $D_1$ in terms of $\Dcap$
 
$$
D_1 = \sum_{l'=1}^{L-1} 2 \Dcap_{l'+\half}
        \left({p_{l'+\half} \over p_{L+\half}} - 1\right)
$$
 
Substituting back in the expression for $D_l$, we obtain
 
$$\eqalignno{
D_l &=  \sum_{l'=1}^{l-1} 2 \Dcap_{l'+\half}
      + \sum_{l'=1}^{L-1} 2 \Dcap_{l'+\half}
          \left({p_{l'+\half} \over p_{L+\half}} - 1\right)\cr
\noalign{\hbox{or}}
D_l &=  \sum_{l'=1}^{L-1}
          2 \left({p_{l'+\half} \over p_{L+\half}}\right) \Dcap_{l'+\half}
      - \sum_{l'=l}^{L-1} 2 \Dcap_{l'+\half}\cr
}$$
 
Defining a column vector of length $L-1$:
$ \Dcapvec = (\Dcap_{1+\half},\ldots,\Dcap_{L-\half})$,
we can write the above relation in matrix form as
 
$$ \Dvec = \Mx[\Dcap,D] \> \Dcapvec
$$
 
where $\Mx[\Dcap,D]$ is an $L \times (L-1)$ matrix defined by
 
$$ \left[\Mx[\Dcap,D]\right]_{l,l'} = \cases{
2\left({p_{l'+\half} \over p_{L+\half}}    \right)     &if $l'<l$;    \cr
2\left({p_{l'+\half} \over p_{L+\half}} - 1\right)     &if $l' \ge l$.\cr}
$$
 
$$  \Mx[\Dcap,D] = \pmatrix{
2\Bigl({p_{1+\half}\over p_{L+\half}}-1\Bigr)&
2\Bigl({p_{2+\half}\over p_{L+\half}}-1\Bigr)& \cdots&
2\Bigl({p_{L-\half}\over p_{L+\half}}-1\Bigr)\cr
2\Bigl({p_{1+\half}\over p_{L+\half}}  \Bigr)&
2\Bigl({p_{2+\half}\over p_{L+\half}}-1\Bigr)& \cdots&
2\Bigl({p_{L-\half}\over p_{L+\half}}-1\Bigr)\cr
2\Bigl({p_{1+\half}\over p_{L+\half}}  \Bigr)&
2\Bigl({p_{2+\half}\over p_{L+\half}}  \Bigr)& \cdots&
2\Bigl({p_{L-\half}\over p_{L+\half}}-1\Bigr)\cr
\vdots&\vdots&\ddots&\vdots\cr
2\Bigl({p_{1+\half}\over p_{L+\half}}  \Bigr)&
2\Bigl({p_{2+\half}\over p_{L+\half}}  \Bigr)& \cdots&
2\Bigl({p_{L-\half}\over p_{L+\half}}  \Bigr)\cr
}$$
 
The above $L \times (L-1)$ matrix has a generalized left inverse which may be
obtained as follows:
 
$$ \Dcap_{l'+\half} = \half D_{l'+1} - \half D_{l'}
  = \sum_{l''=1}^L \half (\delta_{l'',l'+1} - \delta_{l'',l'}) D_{l''}
$$
 
In matrix form, this may be written as
 
$$ \Dcapvec = \Mx[D,\Dcap] \> \Dvec
$$
 
where $\Mx[D,\Dcap]$ is an $(L-1) \times L$ matrix defined by
 
$${\left[ \Mx[D,\Dcap] \right]}_{l',l''} =
  \cases{ -\half & if $l''=l'$;\cr
          +\half & if $l''=l'+1$;\cr
          0      & otherwise.\cr}
$$
 
$$  \Mx[D,\Dcap] = \pmatrix{
-\half& +\half&      0& \cdots&      0&      0\cr
\noalign{\smallskip}
     0& -\half& +\half& \cdots&      0&      0\cr
\vdots& \vdots& \vdots& \ddots& \vdots& \vdots\cr
     0&      0&      0& \cdots& -\half& +\half\cr
}$$
 
$\Mx[D,\Dcap]$ has the following property: $\Mx[D,\Dcap] \> \Mx[\Dcap,D] = 1$.
\medskip
 
Next we write the hydrostatic relation (3d) as
 
$$ \Phicap_{l'+\half} = - C_p \zetacap_{l'+\half} \Thetabar_{l'+\half}
                      = - C_p \sum_{l''=1}^L
\half\,\zetacap_{l'+\half}(\delta_{l'',l'+1}+\delta_{l'',l'})\Theta_{l''}
$$
 
Defining a column vector of length $L-1$:
$\Phicapvec = (\Phicap_{1+\half},\ldots,\Phicap_{L-\half})$,
and a column vector $\Tvec$ of length $L$,
we can write the above equation in matrix form as
 
$$ \Phicapvec = \Mx[\Theta,\Phicap] \> \Tvec
$$
 
where $\Mx[\Theta,\Phicap]$ is an $(L-1) \times L$ matrix defined by
 
$${\left[ \Mx[\Theta,\Phicap] \right]}_{l',l''} =
  \cases{ -\half C_p \zetacap_{l'+\half} & if $l''=l'+1$ or $l''=l'$;\cr
          0                              & otherwise.\cr}
$$
 
$$  \Mx[\Theta,\Phicap] = \pmatrix{
-\half C_p\zetacap_{1+\half}& -\half C_p\zetacap_{1+\half}& 0& \cdots& 0& 0\cr
0& -\half C_p\zetacap_{2+\half}& -\half C_p\zetacap_{2+\half}& \cdots& 0& 0\cr
\vdots& \vdots& \vdots& \ddots& \vdots& \vdots\cr
0& 0& 0& \cdots& -\half C_p\zetacap_{L-\half}& -\half C_p\zetacap_{L-\half}\cr
}$$
 
 
\Section 4. Time integration.
 
\Subsection a. Semi-implicit leap-frog scheme.
 
We will use a leap-frog time-stepping scheme, but the terms associated with
gravity waves will be treated implicitly.  Using superscripts to denote the
time-step, we write (for a generic prognostic quantity $q$)
 
$$ q^{n+1} = q^{n-1} + 2 \Dt \tt[q,n] \eqno (\neweq)
$$
 
where $\Dt$ denotes the time-step, and $(\partial q/\partial t)^n$ denotes an
appropriately defined time-tendency for $q$.
 
Let $\alpha$ (such that $0 \le \alpha \le 1$) denote the ``implicitness'' of
the time-stepping scheme. We add scale-selective ($\deleight$) damping terms
to (3), and define the time-tendencies as follows
 
$$\eqalignno{
\tt[\xivec,n]   &= - \gamma_u \deleight
             \left(\alpha \xivec^{n+1} + (1-\alpha) \xivec^{n-1}\right)
                   - \curlz {\vec {\v H}}_u^n &(\neweq a)\cr
\tt[\Dcapvec,n] &= - \delsq \Mx[\Theta,\Phicap]
             \left(\alpha \Tvec^{n+1} + (1-\alpha) \Tvec^{n-1}\right)
                  - \gamma_u \deleight
             \left(\alpha \Dcapvec^{n+1} + (1-\alpha) \Dcapvec^{n-1}\right)\cr
                &\qquad +\Mx[D,\Dcap](- \div {\vec {\v H}}_u^n
                                      - \delsq{\vec E}^n) &(\cureq b)\cr
\tt[\Tvec,n]    &= - \Mx[\Dcap,H]
               \left(\alpha \Dcapvec^{n+1} + (1-\alpha) \Dcapvec^{n-1}\right)
                   - \gamma_T \deleight
               \left(\alpha \Tvec^{n+1} + (1-\alpha) \Tvec^{n-1}\right)
                   - {\vec H}_{T'}^n \cr
                & &(\cureq c)\cr
}$$
 
where ${\vec \xi}$, ${\vec {\v H}}_u$, ${\vec E}$, and ${\vec H}_{T'}$
are vectors of length $L$, and $\Mx[\Dcap,H] = \Mx[D,H] \> \Mx[\Dcap,D]$.
 
It is convenient to define the ``explicit time-tendencies''
 
$$\eqalignno{
\Xvec_\xi^n &= - \curlz {\vec {\v H}}_u^n - \gamma_u \deleight \xivec^{n-1}\cr
\Xvec_D^n   &= - \div   {\vec {\v H}}_u^n - \delsq {\vec E}^n
               - \delsq \Mx[\Theta,\Phi] \Tvec^{n-1}
               - \gamma_u \deleight \Dvec^{n-1}\cr
\Xvec_T^n   &= - {\vec H}_{T'}^n - \Mx[D,H] \Dvec^{n-1}
               - \gamma_T \deleight \Tvec^{n-1}\cr
}$$
 
where $\Mx[\Theta,\Phi] = \Mx[\Dcap,D] \Mx[\Theta,\Phicap] $.
\smallskip
 
This allows us to rewrite the time-tendency equations (\cureq) as
 
$$\eqalignno{
\tt[\xivec,n]   &= - \gamma_u \deleight \alpha (\xivec^{n+1} - \xivec^{n-1})
                   + \Xvec_\xi^n &(\neweq a)\cr
\tt[\Dcapvec,n] &= - \delsq \Mx[\Theta,\Phicap] \,
                     \alpha (\Tvec^{n+1} - \Tvec^{n-1})
                   - \gamma_u \deleight
                     \alpha (\Dcapvec^{n+1} - \Dcapvec^{n-1})
                   + \Mx[D,\Dcap] \Xvec_D^n \cr
                & &(\cureq b)\cr
\tt[\Tvec,n]    &= - \Mx[\Dcap,H] \alpha (\Dcapvec^{n+1} - \Dcapvec^{n-1})
                   - \gamma_T \deleight
                     \alpha (\Tvec^{n+1} - \Tvec^{n-1})+\Xvec_T^n&(\cureq c)\cr
}$$
 
Assuming that we expand our variables in terms of eigenfunctions of the
$\delsq$ operator, we can define the following ``implicit correction factors''
 
$$ \gamma_u^{corr} \equiv {1 \over 1 + \alpha 2 \Dt \gamma_u \deleight}
$$
$$ \gamma_T^{corr} \equiv {1 \over 1 + \alpha 2 \Dt \gamma_T \deleight}
$$
 
We then use (\releq{-2}) to rewrite (\cureq) as
 
$$\eqalignno{
\tt[\xivec,n]   &=   \gamma_u^{corr} \Xvec_\xi^n &(\neweq a)\cr
\tt[\Dcapvec,n] &= - \gamma_u^{corr} \delsq \Mx[\Theta,\Phicap]
                       (\alpha 2 \Dt) \tt[\Tvec,n]
                    + \gamma_u^{corr} \Mx[D,\Dcap] \Xvec_D^n &(\cureq b)\cr
\tt[\Tvec,n]    &= - \gamma_T^{corr} \Mx[\Dcap,H]
                       (\alpha 2 \Dt) \tt[\Dcapvec,n]
                    + \gamma_T^{corr} \Xvec_T^n &(\cureq c)\cr
}$$
 
We can now solve (\cureq b,c) for the $\Dcap$ tendency
 
$$\eqalignno{
\tt[\Dcapvec,n] &= \quad (\alpha 2 \Dt)^2 \gamma_u^{corr} \gamma_T^{corr}
                    \delsq \Mx[\Theta,\Phicap] \Mx[\Dcap,H] \tt[\Dcapvec,n]\cr
        &\quad - \gamma_u^{corr} \gamma_T^{corr} \delsq \Mx[\Theta,\Phicap]
                   (\alpha 2 \Dt) \Xvec_T^n
                 + \gamma_u^{corr} \Mx[D,\Dcap] \Xvec_D^n\cr
}$$
 
or
 
$$\eqalignno{
\tt[\Dcapvec,n] &= {\left[1 - (\alpha 2 \Dt)^2 \gamma_u^{corr} \gamma_T^{corr}
                     \delsq \Mx[\Theta,\Phicap] \Mx[\Dcap,H] \right]}^{-1}\cr
                &\qquad  \left( - \gamma_u^{corr} \gamma_T^{corr}
                         \delsq \Mx[\Theta,\Phicap] (\alpha 2 \Dt) \Xvec_T^n
                       + \gamma_u^{corr} \Mx[D,\Dcap] \Xvec_D^n\ \right)
                         &(\neweq)\cr
}$$
 
Using the identity $\Mx[D,\Dcap] \> \Mx[\Dcap,D] = 1$,
we can write
 
$$ \Mx[\Theta,\Phicap] = \Mx[D,\Dcap] \Mx[\Dcap,D] \Mx[\Theta,\Phicap]
                       = \Mx[D,\Dcap] \Mx[\Theta,\Phi]
$$
 
which allows us to rewrite (\cureq) as
 
$$\eqalignno{
\tt[\Dvec,n] &= \Mx[\Dcap,D]
                 {\left[1 - (\alpha 2 \Dt)^2 \gamma_u^{corr} \gamma_T^{corr}
                     \delsq \Mx[\Theta,\Phicap] \>
                            \Mx[\Dcap, H] \right]}^{-1} \Mx[D,\Dcap]\cr
             &\quad \left( - \gamma_u^{corr} \gamma_T^{corr}
                    \delsq \Mx[\Theta,\Phi] (\alpha 2 \Dt) \Xvec_T^n
                    + \gamma_u^{corr} \Xvec_D^n\ \right) &(\neweq)\cr
}$$
 
Equations (\releq{-2} a,c), along with (\cureq) constitute the prognostic
equations of the time-stepping scheme.
 
 
\Subsection b. Time-filtering.
 
Since the leap-frog scheme produces a computational mode, a Robert time-filter
(e.g. see Haltiner and Williams, p.~147) is used to damp it out. The leap-frog
scheme as described in the previous section is modified as follows (for a
generic prognostic quantity $q$):
 
$$ q^{n+1} = q_\ast^{n-1} + 2 \Dt \tt[q,n] \eqno (\neweq)
$$
 
where $q_\ast$ denotes the filtered value of $q$ obtained as follows
 
$$ q_\ast^n = (1 - 2\epsilon) q^n + \epsilon(q^{n+1} + q^{n-1})
$$
 
Here $\epsilon$ is a small positive fraction, typically of $O(0.01)$ for
meteorological applications.
 
 
\Section 5. Horizontal discretization.
 
The model uses a truncated series of spherical harmonics to represent the
horizontal variation of a quantity $q(\lambda,\phi,p,t)$. The spherical
harmonics $\Ymn[m,n]$ are defined as
 
$$ \Ymn[m,n](\lambda,\mu) = \Pmn[m,n](\mu) e^{im\lambda}
$$
 
where $\mu \equiv \sin\phi$, and $\Pmn[m,n]$ are the associated Legendre
polynomials. The normalization conditions are
 
$$\eqalignno{
\half\int_{-1}^1 d\mu\> \Pmn[m,n] \Pmn[m',n'] &= \delta_{n,n'}\cr
{1 \over 4\pi} \int_{-1}^1 d\mu \int_0^{2\pi} d\lambda\>
                   \Ymn[m,n]^\ast \Ymn[m',n'] &= \delta_{m,m'} \delta_{n,n'}\cr
}$$
 
Some useful expression for $\Pmn[m,n]$ are
 
$$        \Pmn[0,0] = 1; \qquad \Pmn[0,1] = \sqrt3 \mu;
   \qquad \Pmn[0,2] = {\sqrt5 \over \> 2} (3 \mu^2 - 1)
$$
 
We expand $q$ in terms of the spherical harmonics as
 
$$\eqalignno{
q(\lambda,\mu,p,t) &= \sum_{n=0}^N \; \sum_{m=-\min(n,M)}^{+\min(n,M)}
                      q_{m,n}(p,t) \> \Ymn[m,n](\lambda,\mu) &(\neweq a)\cr
q_{m,n}            &= {1 \over 4\pi} \int_{-1}^1 d\mu \int_0^{2\pi} d\lambda\>
                      \Ymn[m,n]^\ast \> q(\lambda,\mu,p,t) &(\cureq b)\cr
}$$
 
where $M$ and $N$ determine the order of the zonal and the meridional
truncation respectively. The choice $M = N$ corresponds to {\it triangular}
spectral truncation of order $N$. Note that for real $q$, we have the
property that $q_{-m,n} = (-1)^m q_{m,n}^\ast$. Therefore, we only need to
store the values of $q_{m,n}$ for $m \ge 0$.
 
For all the linear terms in the prognostic equations discussed in the previous
section, it is quite straightforward to obtain a separate time-tendency
equation for each spectral coefficient $q_{m,n}$. Since $\Ymn[m,n]$ are
eigenfunctions of the $\delsq$ operator, with eigenvalues $(- n(n+1)/ a^2)$,
it is very easy to compute the horizontal velocity $\uvec$ from $\xi$ and $D$,
by using the streamfunction $\psi$ and velocity potential $\chi$ defined as
follows:
 
$$ \xi = \delsq \psi; \qquad D = \delsq \chi; \qquad
   \uvec = \kvec \times \delh \psi + \delh \chi
$$
 
The quadratic nonlinear products that occur in the prognostic equations are
evaluated using the {\it transform method} (e.g., see Haltiner and Williams,
pp.~193--201). To summarize this method--- consider a quadratic product of the
form $(qr)$. Given the spectral coefficients $q_{m,n}$ and $r_{m,n}$, we wish
to evaluate the spectral coefficients of the product ${(qr)}_{m,n}$. We choose
a $(\lambda_j,\mu_k)$ longitude/sine-latitude grid in physical space, where
$j=1,\ldots,K_1$, and $k=1,\ldots,K_2$. We choose the $K_1$ longitudes to be
equally spaced. The $K_2$ latitudes (referred to as ``gaussian'' latitudes)
are chosen to be gaussian quadrature points for the associated Legendre
polynomials. i.e.
 
$$ \half\int_{-1}^1 d\mu\> \Pmn[m,n] \, \Pmn[m',n'] =
   \half\sum_{k=1}^{K_2} \> G_k \, \Pmn[m,n](\mu_k) \, \Pmn[m',n'](\mu_k),
   \qquad \hbox{for all $n,n' \le N$;}
$$
 
where $G_k$ are the weights associated with each quadrature point $\mu_k$ (cf.
Press et al, pp.~121--126). To avoid aliasing, $K_1$ and $K_2$ must satisfy
the following constraints:
 
$$ K_1 \ge 3M + 1; \qquad K_2 \ge {3N + 1 \over 2}
$$
 
The physical space values $q_{j,k}$ and $r_{j,k}$ at the grid-points
$(\lambda_j,\mu_k)$ are computed using (\cureq a). Then the quadratic term
$(qr)$ is transformed to spectral space as follows:
 
$$ {(qr)}_{m,n} = {1 \over 2K_1} \sum_{k=1}^{K_2} \> G_k \>
       \sum_{j=1}^{K_1} \Ymn[m,n]^\ast(\lambda_j,\mu_k) \> q_{j,k} \> r_{j,k}
$$
 
Thus the transforms between spectral and physical representations essentially
involve a discrete Legendre transform combined with a discrete Fourier
transform.
 
 
\Section 6. Miscellaneous details.
 
\Subsection a. Passive tracers.
 
A conserved passive tracer $q$ is assumed to be governed by the equation
 
$$ \pd[q,t] = - \div(q\uvec) -  \pd[(\omega q), p]
$$
 
It is discretised in a manner very similar to the $\Theta$ equation, and the
prognostic equation is written as
 
$$ \tt[q_l,n] = \gamma_T^{corr} \left[ - \div(q_l^n\uvec_l^n)
                  -  \vdiv{V_q^n}  - \gamma_T \deleight q_l^{n-1} \right]
$$
 
 
\Subsection b. Choice of reference stratification.
 
For numerical stability, $\Theta^R(p)$ should preferably be chosen such that
the reference values of static stability $\{-(\partial\Theta^R/\partial p)\}$
are typically larger than the values of static stability likely to be
encountered during time-integrations with the model (Simmons~et~al, 1978).
Before we define $\Theta^R(p)$, we define a ``standard'' vertical profile
$\Theta^S(p)$. This ``standard'' profile may be some representation of the
globally averaged vertical stratification such as the U.S. Standard
Atmosphere.  We use $\Theta^S$ to construct $\Theta^R$ as follows:
 
$$ \Theta^R_L = \Theta^S_L; \qquad
   \Thetacap^R_{l+\half} = s^R \, \Thetacap^S_{l+\half}
$$
 
i.e. $\Theta^R$ coincides with $\Theta^S$ at the surface, but the increase of
$\Theta^R$ with decreasing pressure is steeper than that of $\Theta^S$ by a
``steepness'' factor $s^R$. One would typically choose $s^R > 1$.
 
The standard profile $\Theta^S$ is also used to define a standard height $z^S$
as follows:
 
$$ g z^S_L =   C_p (1-\zeta_L) \Theta^S_l; \qquad
   g z^S_l = 2 C_p \zetacap_{l+\half} \Thetabar^S_{l+\half} + g z_{l+1}
$$
 
The standard height values may be useful for the purpose of displaying
diagnostics.
 
 
\Subsection c. Forcing/damping terms.
 
We had previously introduced the scale-selective $\deleight$ damping term
during our discussion of the semi-implicit scheme. Now we discuss that and
other damping terms in some more detail. We redefine the ``explicit
time-tendencies'' as follows:
 
$$\eqalignno{
X_{\xi,l}^n    &= \cdots - \gamma(\delsq+{2\over a^2})^4 \xi_l^{n-1}
                         - \eta_{u,l} \xi_l^{n-1}
                         + \curlz \vdiv{{\v V}_{\nu u}^{n-1}} \cr
X_{D,l}^n      &= \cdots - \gamma(\delsq+{2\over a^2})^4 D_l^{n-1}
                         - \eta_{u,l} D_l^{n-1}
                         + \div   \vdiv{{\v V}_{\nu u}^{n-1}} \cr
X_{\Theta,l}^n &= \cdots - \gamma(\delsq)^4 \Theta_l^{n-1}
                         - \eta_{T,l} (\Theta_l^{n-1} - \Theta^M_l)\cr
}$$
 
where ($\cdots$) denotes the adiabatic explicit tendencies, i.e., without any
damping at all; $\gamma$ is the scale-selective $\deleight$ damping
coefficient (usually expressed in units of $a^8/sec$), $\eta_u$ is a pressure-
dependent Rayleigh friction coefficient, $\eta_T$ is a pressure-dependent
Newtonian cooling coefficient. $\Theta^M$ is a specified ``mean'' potential
temperature distribution that the Newtonian cooling relaxes $\Theta$ back to.
 
Note that there is a subtle difference in the way that the $\deleight$ damping
acts on $\xi$ or $D$ as compared to $\Theta$, i.e., we have made the following
substitutions: $\gamma_u \deleight \to \gamma (\delsq+2a^{-2})^4$ and $\gamma_T
\deleight \to \gamma(\delsq)^4$. The above scheme, which is motivated by the
properties of horizontal viscous mixing on a sphere, ensures that a state of
solid-body rotation is not affected by the horizontal scale-selective damping.
Also note that all the damping terms involve values of the prognostic quantity
at time step $n-1$. The damping terms are not treated using a leap-frog scheme
because it would lead to numerical instabilities.
 
${\v V}_{\nu u}$ is the vertical viscous stress, defined as follows:
 
$$\eqalignno{
{\v V}_{\nu u,l+\half} &= \nu_{l+\half} {\uvec_{l+1} - \uvec_l \over
                                         \half \Dp_{l+1} + \half \Dp_l} \cr
{\v V}_{\nu u,\half}   &= 0 \cr
{\v V}_{\nu u,L+\half} &= - \Dp_L \>  \eta_{sd} \> \uvec \cr
\nu_{l+\half}          &= \nu_0 {\left(\pd[p,z]\right)}_l^2
                        = \nu_0 {\left({\half \Dp_{l+1} + \half \Dp_l \over
                                 z^S_{l+1} - z^S_l}\right)}^2 \cr
}$$
 
where $\eta_{sd}$ is the surface drag coefficient, and $\nu_0$ is a
height-independent ``vertical'' kinematic viscosity.
 
 
\Section 7. Fortran implementation.
 
\Subsection a. Overview.
 
The model described so far has been implemented as set of Fortran77
subroutines. The code has been written with simplicity, rather than
efficiency, in mind. A simple storage scheme is used for the spectral
coefficients, which leads to about 50\% redundancy in memory usage.  The
execution speed should not be too bad, because an effort has been made to
ensure that the major ``innermost'' loops of the spectral transform routines
automatically vectorize on vector-processors. One important feature of the
time-marching routines in module {\it prognos} is that they carry out spectral
transforms over all the latitudes at the same time. This keeps the code simple
and efficient, but the table of Legendre polynomials needed for this procedure
becomes rather large at high orders of spectral truncation.
 
The fortran source code is spread out over the following files:
 
{\it prognos.F, splib.F, mcons.h, mgrid.h, tmarch.h, spcons.h, spgrid.h,
sppoly.h, spfftb.h, {\rm and} ftable.h.}
 
The two ``{\sl .F}'' files contain Fortran source code interspersed with C
preprocessor directives. (On UNIX systems, the {\sl f77} command automatically
invokes the C preprocessor on such files). There are two source modules: {\it
prognos} and {\it splib}. The module {\it splib} deals exclusively with
spectral transforms, and is self-contained. The module {\it prognos}
implements the time-stepping scheme for the model, and uses some of the
routines available in module {\it splib}. {\sl It also uses a subroutine from
the NAG library to invert the implicit correction matrix.}
 
The ``{\sl .h}'' files are ``include'' files used by the ``{\sl .F}'' modules.
There are two types of include files--- C-parameter include files and Fortran
declarative include files. The former contain definitions of C preprocessor
parameters, and need to be included only once in the whole file, preferably at
the very beginning of the file. The latter type of include files usually
contain declarations of Fortran COMMON variables, and would need need to be
included in the declarative part of every Fortran function/subroutine that has
occasion to use them.
 
There are two C-parameter include files: {\it mcons.h} and {\it spcons.h}.
The file {\it mcons.h} contains important C preprocessor definitions for
module {\it prognos}. Therefore, any program that uses routines from module
{\it prognos} should begin with the following line---
 
{\sl \#include "mcons.h"}
 
Similarly, any program using routines from module {\it splib} should
``include'' the file {\it spcons.h} at the very beginning, unless {\it
mcons.h} has already been included. (Including {\it mcons.h} automatically
causes {\it spcons.h} to be included.)
 
The remaining include files contain Fortran declarations. The include file
{\it mgrid.h} contains several useful COMMON variables of the primitive
equation model, which are described in the preface to module {\it prognos}.
Similarly, the include file {\it spgrid.h} contains several useful variables
related to the horizontal truncation and the spectral transforms, which are
described in the preface to module {\it splib}. (Including {\it mgrid.h}
automatically causes {\it spgrid.h} to be included.)
 
The declarative include file {\it tmarch.h} contains COMMON variables dealing
with the time-stepping scheme, but they would rarely be needed by other
programs. The include files {\it sppoly.h, spfftb.h,} and {\it ftable.h} are
meant primarily for internal use by module {\it splib}.
 
Appendix B contains a Fortran to Symbols ``dictionary,'' which lists the
symbolic equivalents of most Fortran variable names.
 
\Subsection b. Module {\rm splib}.
 
Module {\it splib} contains numerous routines dealing with spectral
transforms. Only the most commonly used routines, viz., the ones used by
module {\it prognos}, are described below.  To use these routines, the
following conventions must be used:
 
\item{i)} A spectral space variable $q_{m,n}$ should be declared as
 
COMPLEX QSP(0:M1MAX, 0:N1MAX)
 
\item{ii)} A physical space variable $q_{j,k}$ should be declared as
 
REAL QPH(K1MAX, K2MAX)
 
 
The major subroutines of module {\it splib} may be described as follows (where
the variables in parantheses denote arguments to the subroutine, $q, r$ denote
generic scalars, $\uvec = u{\v i} + v{\v j}$ denotes a generic vector field,
and ``$\gets$'' denotes the assignment operator):
 
\smallskip
\litem SPINI$(M,N,a):$ Initializes the spectral truncation order to $(M,N)$ and
the planetary radius to $a$ \smallskip
 
\litem ZEROSP$(q_{m,n},N): q_{m,n} \gets 0$ \smallskip
 
\litem SPCOPY$(r_{m,n},q_{m,n},N): r_{m,n} \gets q_{m,n}$ \smallskip
 
\litem DELSQ$(r_{m,n},q_{m,n},N): r_{m,n} \gets \delsq q_{m,n}$ \smallskip
 
\litem IDELSQ$(r_{m,n},q_{m,n},N): r_{m,n} \gets \delh^{-2} q_{m,n}$ \smallskip
 
\litem PHYSIC$(r_{j,k},q_{m,n}): r_{j,k} \gets (q)_{j,k}$ \smallskip
 
\litem HVELOC$(u_{j,k},v_{j,k},\psi_{m,n},\chi_{m,n}):
 \uvec_{j,k} \gets {(\kvec \times \delh \psi + \delh \chi)}_{j,k}$ \smallskip
 
\litem SPECTR($r_{m,n},q_{j,k}): r_{m,n} \gets r_{m,n} + (q)_{m,n}$ \smallskip
 
\litem DIVERG$(r_{m,n},u_{j,k},v_{j,k}):
               r_{m,n} \gets r_{m,n} + (\div   \uvec)_{m,n}$ \smallskip
 
\litem CURLZ$(r_{m,n},u_{j,k},v_{j,k}):
               r_{m,n} \gets r_{m,n} + (\curlz \uvec)_{m,n}$ \smallskip
 
In short, SPINI initializes all spectral transform operations, SPCOPY copies
spectral representations, and DELSQ\slash IDELSQ apply the
Laplacian/inverse-Laplacian operators respectively to spectral
representations. PHYSIC converts spectral representations to physical
representations, and HVELOC computes the physical velocity, given the spectral
streamfunction and velocity potential. The last three routines (SPECTR,
DIVERG, CURLZ) convert from physical to spectral representation, and {\sl
accumulate}. ZEROSP should be used to set the spectral representation to zero
before accumulation, if necessary. SPECTR converts from physical to spectral
representation, DIVERG converts the divergence of a physical space vector
field to spectral space, and CURLZ does the same for the $\curlz$ of a
physical space vector field. The divergence and $\curlz$ are computed through
integration by parts (cf. Haltiner and Williams, p.~194).
 
Module {\it splib} also provides an error exit routine, SPERR, which takes two
string arguments--- the name of the calling subroutine and an error message.
SPERR simply displays the two strings and then aborts the program.
 
(In case you are wondering as to why some of the subroutines take $N$ as an
argument, when its value has already been specified through a call to SPINI---
those subroutines can also take $N+1$ as an argument, to handle the
exceptional case of the spectral representation of a vector component, which
is of one meridional order higher than the spectral representation of a
scalar. But for ordinary usage, $N$ can simply be thought of as being a
redundant argument)
 
 
\Subsection c. Module {\rm prognos}.
 
Module {\it prognos} contains subroutines for initializing the planetary
parameters and the vertical resolution. It also contains the time-marching and
time-filtering routines. Before we describe these routines, it is useful to
define the {\it composite level variable} ${\v P}_l = (P_{1,l}, P_{2,l},
P_{3,l}, \ldots) \equiv (\xi_l, D_l, \Theta_l, \ldots)$, where $(\ldots)$
denotes the concentrations of passive traces (if any). i.e. We then use the
notation ${\vec {\v P}}$ to denote the spectral representation of ${\v P}$ at
all the levels taken together. To use these routines, the following
conventions must be used:
 
\item{i)} A spectral space variable $q_{m,n,l}$ should be declared as
 
COMPLEX QSP(0:M1MAX, 0:N1MAX, L1MAX)
 
\item{ii)} A physical space variable $q_{j,k,l}$ should be declared as
 
REAL QPH(K1MAX, K2MAX, L1MAX)
 
\item{iii)} A spectral composite level variable $P_{m,n,i,l}$ should be
declared as
 
COMPLEX PSP(0:M1MAX, 0:N1MAX, NPGQ, L1MAX)
 
where $i=1$ denotes $\xi$, $i=2$ denotes $D$, $i=3$ denotes $\Theta$, and
values of $i>3$ denote passive tracers (if any); NPGQ $\equiv 3+$ NTRACE,
where NTRACE is the number of passive tracers.
 
The major subroutines of module {\it prognos} may be described as follows
(where the variables in parantheses denote arguments to the subroutine, ${\vec
{\v P}}, {\vec {\v Q}}$ denote generic composite level variables):
 
\litem PLINI$(\Omega,R,C_p,q):$ Initializes the planetary parameters
 
\litem VERINI$(L,\Dp_l,\Theta^S_l,s^R,\gamma,\Theta^M_l,\eta_{T,l},\eta_{u,l},
\eta_{sd},\nu_0):$ Initializes the number of levels, level thicknesses,
standard temperature profile, and forcing/damping details
 
\litem DDTINI$(\Dt,\alpha):$ Initializes the semi-implicit time-stepping
scheme
 
\litem DDTPGQ$({\vec {\v Q}}, {\vec {\v P}}^{n-1}, {\vec {\v P}}^n):
{\vec {\v Q}} \gets \tt[{\vec {\v P}},n]_{ad.}$. i.e., DDTPGQ computes the
``adiabatic'' time-tendency (We put ``'' around {\sl adiabatic} because the
effects of scale-selective damping, which is the only damping term that is
treated semi-implicitly, are also included.)
 
\litem DDTFRC$({\vec {\v Q}}, {\vec {\v P}}^{n-1}, {\vec {\v P}}^n):
{\vec {\v Q}} \gets {\vec {\v Q}} + (\hbox{forcing terms})$. i.e. DDTFRC adds
the explicit forcing\slash damping terms to the time-tendency ${\vec {\v Q}}$.
 
\litem DDTIMP$({\vec {\v Q}}):{\vec {\v Q}} \gets (\hbox{implicitly corrected }
{\vec {\v Q}})$. i.e. DDTIMP makes implicit corrections to the explicit
time-tendency ${\vec {\v Q}}$.
 
\litem ROBFIL$(q_{m,n}^{n-1},q_{m,n}^n,q_{m,n}^{n+1},\epsilon): q_{m,n}^n
\gets (1 - 2\epsilon)q_{m,n}^n+\epsilon(q_{m,n}^{n+1}+q_{m,n}^{n-1})$. i.e.,
ROBFIL applies the Robert time-filter.
 
 
\Subsection d. Compilation and customization.
 
Both {\it prognos} and {\it splib} contain Fortran subroutines/functions, but
no MAIN programs. They may be concatenated together with a main program to
form an executable model. The concatenated source code may be compiled with an
{\sl f77} command on UNIX systems (or {\sl cf77} on UNICOS). To customize the
model for different resolutions etc., many of the important C preprocessor
parameters may be re-defined in the compilation command line itself. The
important customization parameters are defined below:
 
\litem{DPRECISION:} This parameter, if defined, indicates that the default
floating point precision on the machine for REAL variables is equivalent to
REAL*8. This parameter is needed because some of the initializations need to
be carried out at double precision on 32-bit machines. This could be avoided
on 64-bit machines (such as the Cray) by defining this parameter.
 
\litem{UDEF:} The modules {\it splib} and {\it prognos} can be compiled with
the {\sl f77 -u} option on UNIX systems, because all variables and functions
are declared explicitly before being used. If the MAIN program also follows
this convention, the {\sl-u} option could be used to detect bugs arising from
the use of undefined variables. But a small number of ``dummy'' declarations
had to be introduced to be able to do this on some UNIX f77 compilers, perhaps
because of a bug in the f77 compiler. These dummy declarations generate errors
without the {\sl-u} option. The C preprocessor parameter UDEF is provided so
that these these ``dummy'' declarations may be enabled only when needed.
Therefore, the UDEF option should be defined if and only if the {\sl f77 -u}
option is used to compile the program.
 
\litem{N1MAX:} This is the maximum order of triangular truncation (i.e. order
of Legendre polynomials). The default value is 21.
 
\litem{M1MAX:} This is the maximum value of zonal wavenumber resolved by the
truncation. The default value is N1MAX.
 
\litem{K1MAX:} This is the number of equally spaced longitudes in the
transform grid [should be the lowest integral power of 2 $\ge$ (3*N1MAX+1)].
The default value is 64.
 
\litem{K2MAX:} This is the number of gaussian latitudes in the transform grid
[should be the lowest odd integer $\ge$ (3*N1MAX+1)/2]. The default value is
33. (K2MAX is always chosen to be odd, so that the equator can be one of the
gaussian latitudes)
 
\noindent [Other typical choices of (N1MAX, M1MAX, K1MAX, K2MAX) could be (10,
10, 32, 17) or (21, 21, 64, 33) or (42, 42, 128, 65) or (85, 85, 256, 129)]
 
\litem{L1MAX:} This is the maximum number of pressure-levels $(\ge2)$. The
default value is 20.
 
\litem{FIXTRUC:} This option, if defined, fixes the horizontal/vertical
resolution of the model. i.e. The number of levels and the spectral truncation
are fixed. By default, this option is not defined, and the resolution is
allowed to vary within the limits of available storage. But by defining
FIXTRUNC, one can force the counts of many DO loops to become constants, and
this may help improve speed of execution (but there are no guarantees).
 
\litem{NTRACE:} This is the number of passive tracers $(\ge0)$. The default
value is 0.
 
\litem{MLOW:} Cut-off value of zonal wavenumber $m$ that denotes ``low order
zonal truncation''. For truncations with very few zonal wavenumbers ($M \le$
MLOW), transforms can be performed more efficiently by reversing the order of
certain nested DO loops at the heart of the Legendre transform. But the choice
of MLOW would be very much machine dependent. The default value is 4.
 
\noindent e.g. One may use the commands
 
{\sl cat main.F prognos.F splib.F $>$ model.F}
 
{\sl f77 -O -DL1MAX=5 -DN1MAX=42 -DK1MAX=128 -DK2MAX=65 model.F -lnag}
 
to compile a version of the model with upto 5 pressure levels and horizontal
truncation upto T42, on a Sun workstation. The first line of the main program
{\sl main.F} should be {\sl \#include "mcons.h"}. (If the NAG library is not
available, it may be substituted with routines from the LINPACK library. A
module called {\sl linpack.F}, containing selected LINPACK routines, is
provided to simulate the NAG routines used by module {\sl prognos}. This
module may be compiled separately to form {\sl linpack.o}, and the {\sl -lnag}
option on the above compile command line may simply be substituted with {\sl
linpack.o}.)
 
\noindent One may use the commands
 
{\sl cat main.F prognos.F splib.F $>$ model.F}
 
{\sl cf77 -DDPRECISION -DFIXTRUNC -DM1MAX=0 -DK1MAX=1 model.F -lnag}
 
to compile a $N=21$ axisymmetric version of the model, with 20 pressure levels
and fixed truncation, on a Cray running UNICOS.
 
 
\Subsection e. Initialization and time-marching.
 
First, the horizontal truncation should be initialized through a call to
routine SPINI. Then the planetary parameters should be initialized by calling
PLINI. This should be followed by a call to VERINI to initialize the vertical
resolution, and the forcing/damping parameters.
 
The time-marching sequence is initialized by a call to DDTINI. If one already
has available the values of prognostic variables $({\vec {\v P}}^{n-1},{\vec
{\v P}}^n)$ at two consecutive time-steps, one may use the call
DDTINI$(\Dt,\alpha)$ to initialize the semi-implicit leap-frog time-marching,
with time-step $\Dt$.  Then for each time-step, one would need to execute the
following sequence of instructions:
 
CALL DDTPGQ$({\vec {\v Q}}, {\vec {\v P}}^{n-1}, {\vec {\v P}}^n)$
 
CALL DDTFRC$({\vec {\v Q}}, {\vec {\v P}}^{n-1}, {\vec {\v P}}^n)$
 
$\langle$ {\sl Other explicit forcing terms may be added to ${\vec {\v Q}}$ at
this point} $\rangle$
 
CALL DDTIMP$({\vec {\v Q}})$
 
${\vec {\v P}}^{n+1} \gets {\vec {\v P}}^{n-1} + 2\Dt\, {\vec {\v Q}}$
 
CALL ROBFIL$(q_{m,n}^{n-1},q_{m,n}^n,q_{m,n}^{n+1},\epsilon)$ {\sl for each
constituent $q$ in ${\vec {\v P}}^n$ }
 
But at the very beginning of model integrations, one may have only the initial
condition ${\vec {\v P}}^0$ available. In that case, one may ``cheat'' the
time-marching routines by initializing with the call
DDTINI$(\half\Dt,\alpha)$. Then, for the first time-step only, one may execute
the following sequence of instructions:
 
CALL DDTPGQ$({\vec {\v Q}}, {\vec {\v P}}^0, {\vec {\v P}}^0)$
 
CALL DDTFRC$({\vec {\v Q}}, {\vec {\v P}}^0, {\vec {\v P}}^0)$
 
$\langle$ {\sl Other explicit forcing terms may be added to ${\vec {\v Q}}$ at
this point} $\rangle$
 
CALL DDTIMP$({\vec {\v Q}})$
 
${\vec {\v P}}^1 \gets {\vec {\v P}}^0 + \Dt\, {\vec {\v Q}}$
 
\vfill\eject
 
 
\Appendix A. Symbols and notation.
 
 
\Subsection a. Symbols.
 
(Note: The word ``horizontal'' is used here to denote motions along isobaric
surfaces, which are assumed to deviate only very little from the true
horizontal)
 
\v i, \v j, \v k are the local eastward, northward, and upward unit vectors.
S.I. units are used to express almost all the quantities, except for $\gamma$,
and except where otherwise noted.
 
\halign{\indent#\hfil&\quad\hfil#\hfil\quad&\quad#\hfil\cr
$t$ &:& time (in $s$)\cr
$\lambda$ &:& longitude (in $rad$)\cr
$\phi$ &:& latitude (in $rad$)\cr
$\mu = \sin\phi$ &:& sine-latitude\cr
$z$ &:& height (in $m$)\cr
$p$ &:& pressure (in $Pa$)\cr
$\uvec = u\v i+v\v j$ &:& horizontal velocity (in $m/s$)\cr
$u$ &:& zonal velocity (in $m/s$)\cr
$v$ &:& meridional velocity (in $m/s$)\cr
$\omega = dp/dt$ &:& ``pressure velocity'' (in $Pa/s$)\cr
$T$ &:& temperature (in $K$)\cr
$a$ &:& planetary radius (in $m$)\cr
$\Omega$ &:& angular velocity of planetary rotation (in $1/s$)\cr
$f = 2\Omega \sin\phi$ &:& the coriolis parameter (in $1/s$)\cr
$g$ &:& gravitational acceleration at planetary surface (in $m^2/s$)\cr
$\Phi = gz$ &:& geopotential height (in $m^2/s^2$)\cr
$R$&:& gas constant (per unit mass) for dry air (in $J/\{kgK\}$)\cr
$C_p$&:& specific heat (per unit mass) at constant pressure (in $J/\{kgK\}$)\cr
$C_v$&:& specific heat (per unit mass) at constant volume (in $J/\{kgK\}$)\cr
$\kappa = C_p/C_v$ &:& a ratio of specific heats\cr
$p_s$ &:& the (constant) reference surface pressure (in $Pa$)\cr
$\zeta = {(p/p_s)}^\kappa$ &:& an auxiliary vertical coordinate\cr
$\Theta = T{(p_s/p)}^\kappa$ &:& potential temperature (in $K$)\cr
$s^R$ &:& steepness factor for $\Theta^R$\cr
$\Theta^R$ &:& reference potential temperature profile (in $K$)\cr
$\Theta^S$ &:& standard potential temperature profile (in $K$)\cr
$z^S$ &:& standard height values at pressure levels (in $m$)\cr
$\Theta^M$ &:& ``mean'' potential temperature profile (in $K$)\cr
$\xi = \curlz\uvec$ &:& relative vorticity (in $1/s$)\cr
$D = \div\uvec$ &:& divergence (in $1/s$)\cr
${\v P} = (\xi,D,\Theta,\ldots)$ &:& composite level variable\cr
$\psi = \nabla_H^{-2} \xi$ &:& streamfunction (in $m^2/s$)\cr
$\chi = \nabla_H^{-2} D$ &:& velocity potential (in $m^2/s$)\cr
$E = \half \uvec^2$ &:& kinetic energy (per unit mass, in $m^2/s^2$)\cr
$\alpha$ &:& ``implicitness'' factor $(0 \le \alpha \le 1)$\cr
$\epsilon$ &:& Robert filter factor\cr
$m$ &:& zonal wavenumber\cr
$n$ &:& meridional index of spherical harmonic $(\mid m \mid \le n)$\cr
$M$ &:& zonal wavenumber truncation order $(\mid m \mid \le M)$\cr
$N$ &:& meridional truncation order $(n \le N)$\cr
$\Pmn[m,n](\phi)$ &:& associated Legendre polynomial\cr
$\Ymn[m,n](\lambda,\phi) = \Pmn[m,n]e^{im\lambda}$ &:& spherical harmonic\cr
$K_1$ &:& number of longitudes in the transform grid\cr
$K_2$ &:& number of gaussian-latitudes in the transform grid\cr
$G_k$ &:& gaussian quadrature weight at $\phi = \phi_k$\cr
$\gamma$ &:& $\deleight$ damping coefficient (in units of $a^8/s$)\cr
$\eta_u$ &:& pressure-dependent Rayleigh friction coefficient (in $1/s$)\cr
$\eta_T$ &:& pressure-dependent Newtonian cooling coefficient (in $1/s$)\cr
$\eta_{sd}$ &:& surface friction coefficient (in $1/s$)\cr
$\nu_0$ &:& height-independent ``vertical'' kinematic viscosity (in $m^2/s$)\cr
}
 
\Subsection b. Operators.
 
Vector quantities are shown in boldface. e.g. $\v V = (V_\lambda,V_\phi)$
 
$$\eqalignno{
\delh  &=  \v i{1 \over a\cos\phi}{\partial\over\partial\lambda} +
           \v j{1 \over a        }{\partial\over\partial\phi}\cr
\delsq &= {1 \over a^2\cos^2\phi}{\partial^2\over{\partial\lambda}^2} +
          {1 \over a^2\cos\phi  }{\partial\over\partial\phi}\cos\phi
                                 {\partial\over\partial\phi}\cr
\div  \v V &= {1 \over a\cos\phi}\pd[V_\lambda,\lambda] +
          {1 \over a\cos\phi}{\partial\over\partial\phi}\cos\phi V_\phi\cr
\curlz\v V &= (\kvec\times\delh)\cdot\v V
        = {1 \over a\cos\phi}\pd[V_\phi,\lambda] -
          {1 \over a\cos\phi}{\partial\over\partial\phi}\cos\phi V_\lambda\cr
}$$
 
\vfill\eject
 
 
\Appendix B. Fortran to Symbols dictionary.
 
 
\halign{\indent#\hfil$\to$&\quad$#$\hfil\cr
\omit&\cr
A0     & a \cr
A0INV  & a^{-1} \cr
A0Q    & a^2 \cr
A0QINV & a^{-2} \cr
\omit&\cr
CIM    & i\,m \cr
COSINV & 1/\cos\phi_k \cr
COSPHI & \cos\phi_k \cr
CP     & C_p \cr
\omit&\cr
D2DCAP & \Mx[D,\Dcap] \cr
D2TT   & - \Mx[D,H] = - \Mx[\omega,H] \> \Mx[D,\omega] \cr
D2W    & \Mx[D,\omega] \cr
DCAP2D & \Mx[\Dcap,D] \cr
DEL8DF & \gamma \cr
DP     & \Dp_l \cr
DPGQSP & \partial{\vec {\v P}}/\partial t \cr
DT     & \Dt \cr
DTFAC  & \alpha 2\Dt \cr
\omit&\cr
F0     & 2\Omega \cr
FSP01  & 2\Omega / \sqrt3 \cr
\omit&\cr
G      & \half G_k \qquad \langle \hbox{note factor of } \half \rangle \cr
G0     & g \cr
\omit&\cr
HDP    & \half \Dp_l \cr
\omit&\cr
IMPCOR & \hbox{implicit correction matrix} \cr
IMPFAC & \alpha \cr
IMPLCT &  \alpha \cr
\omit&\cr
J      & j \cr
JDIV   & i=2 \qquad (D) \cr
JPOT   & i=3 \qquad (\Theta) \cr
JVOR   & i=1 \qquad (\xi) \cr
\omit&\cr
K      & k \cr
K1     & K_1 \cr
K1MAX  & \hbox{maximum value of } K_1 \cr
K2     & K_2 \cr
K2MAX  & \hbox{maximum value of } K_2 \cr
KAPPA  & \kappa \cr
\omit&\cr
L      & l \cr
L1     & L \cr
L1MAX  & \hbox{maximum value of } L \cr
LAMBDA & \lambda_j \cr
\omit&\cr
M      & m \cr
M1     & M \cr
M1MAX  & \hbox{maximum value of } M \cr
MU     & \mu_k \cr
\omit&\cr
N      & n \cr
N1     & N \cr
N1MAX  & \hbox{maximum value of } N \cr
N2     & N+1 \cr
NLEV   & L \cr
NNT2DT & \Mx[\Dcap,D] \> \Mx[\Theta,\Phicap] \cr
\omit&\cr
OMEGA0 & \Omega \cr
\omit&\cr
PGQSP0 & {\vec {\v P}}^{n-1} \cr
PGQSP1 & {\vec {\v P}}^n \cr
PHI    & \phi_k \cr
PHLV   & p_{l+\half} \cr
PKCHLV & \zetacap_{l+\half} \cr
PKLV   & \zeta_l \cr
PLV    & p_l \cr
PSURF  & p_s \cr
PTHICK & \Dp_l \cr
\omit&\cr
QHDAMP & \gamma \cr
\omit&\cr
RADIUS & a \cr
RGAS   & R \cr
ROBFAC & \epsilon \cr
\omit&\cr
SFRIC  & \eta_{sd} \cr
\omit&\cr
T2GPCP & \Mx[\Theta,\Phicap] \cr
TD8COR & 1/\left\{1 + \gamma (\delsq)^4 \right\} \cr
TD8FAC & \gamma (\delsq)^4 \cr
TDAMP  & \eta_{T,l} \cr
TMNLV  & \Theta^M_{0,n,l} \cr
TRADEQ & \Theta^M_{k,l} \cr
TREFLV & \Theta^R_l \cr
TRFFAC & s^R \cr
TRLXLV & \eta_{T,l} \cr
TSTDLV & \Theta^S_l \cr
TSTEEP & s^R \cr
TZSTD  & \Theta^S_l \cr
\omit&\cr
UD8COR & 1/\left\{1 + \gamma (\delsq+2a^{-2})^4 \right\} \cr
UD8FAC & \gamma (\delsq+2a^{-2})^4 \cr
UDAMP  & \eta_{u,l} \cr
URLXLV & \eta_{u,l} \cr
USDRAG & \eta_{sd} \cr
\omit&\cr
VERVIS & \nu_0 \cr
VVISC  & \nu_l \cr
\omit&\cr
W2TT   & - \Mx[\omega,H] \cr
\omit&\cr
ZSTDLV & z^S_l \cr
\omit&\cr
}
 
\vfill\eject
 
 
\Appendix C. Modifications to allow forcing at bottom boundary.
 
 
Instead of setting the vertically integrated divergence to zero, one could
choose to specify the geopotential $\Phi$ near the bottom boundary. This
boundary condition may be appropriate for, say, a model of the middle
atmosphere, where the lower boundary forcing due to the troposphere is
specified. We choose to set the pressure-velocity $\omega$ to zero at the
upper boundary, but at the lower boundary we specify the following conditions:
 
$$ {\overline \uvec}_{L+\half} = \uvec_L, \quad
    {\Thetabar'}_{L+\half} = {\Theta'}_L, \quad
    \Phi_L = C_p \zetacap_{L+\half} \Theta_L + \Phi_s(\lambda,\phi,t)
$$
 
where $\zetacap_{L+\half} \equiv (1-\zeta_L)$, and $\Phi_s$ is some specified
function.
 
We also assume that $\Thetacap^R_{L+\half}$ has some specified value. For
convenience, one may even choose to set $\Thetacap^R_{L+\half} =
\Thetacap^R_{L-\half}$. We re-define $\Wvec$ to be vector of length $L$, i.e.,
$\Wvec = (\omega_{1+\half},\ldots,\omega_{L+\half})$. This allows us to
express $\Hvec$ as
 
$$ \Hvec = \Mx[\omega,H] \> \Wvec
$$
 
where $\Mx[\omega,H]$ is re-defined to be an $L \times L$ matrix with the
following structure:
 
$$  \Mx[\omega,H] = \pmatrix{
\Thetacap^R_{1+\half}\over\Dp_1&0                              &\cdots&0&0&0\cr
\Thetacap^R_{1+\half}\over\Dp_2&\Thetacap^R_{2+\half}\over\Dp_2&\cdots&0&0&0\cr
\vdots&\vdots&\ddots&\vdots&\vdots&\vdots\cr
0&0&\cdots
&\Thetacap^R_{L-1-\half}\over\Dp_{L-1}&\Thetacap^R_{L-\half}\over\Dp_{L-1}&0\cr
0&0&\cdots&0&\Thetacap^R_{L-\half}\over\Dp_L&\Thetacap^R_{L+\half}\over\Dp_L\cr
}$$
 
We also re-define $\Mx[D,\omega]$ to be an $L \times L$ matrix. (The extension
is straightforward.) We then define the compound matrix
 
$$\Mx[D,H] = \Mx[\omega,H] \> \Mx[D,\omega]
$$
 
We also define a new quantity $\Phitilde_l \equiv \Phi_l - \Phi_s$. We then
write
 
$$\eqalignno{
\Phitilde_l &= - 2\Phicap_{l+\half} + \Phitilde_{l+1}
             = - \sum_{l'=l}^{L-1}2\Phicap_{l'+\half}
               + C_p \zetacap_{L+\half} \Theta_L\cr
            &=   \sum_{l'=l}^{L-1}2C_p\zetacap_{l'+\half}\Thetabar_{l'+\half}
               + C_p \zetacap_{L+\half} \Theta_L
             =   C_p \zetacap_{l+\half} \Theta_l + \sum_{l'=l+1}^L
                   C_p (\zetacap_{l'-\half}+\zetacap_{l'+\half})\Theta_{l'}\cr
}$$
 
Defining a column vector of length $L$:
$\Phitildevec = (\Phitilde_1,\ldots,\Phitilde_L)$,
we can write the above equation in matrix form as
 
$$ \Phitildevec = \Mx[\Theta,\Phitilde] \> \Tvec
$$
 
where $\Mx[\Theta,\Phitilde]$ is an $L \times L$ matrix defined by
 
$${\left[ \Mx[\Theta,\Phitilde] \right]}_{l,l'} =
  \cases{ C_p (\zetacap_{l'-\half}+\zetacap_{l'+\half}) & if $l'>l$;\cr
          C_p  \zetacap_{l'+\half}                      & if $l'=l$;\cr
          0                                             & otherwise.\cr}
$$
 
$$  \Mx[\Theta,\Phitilde] = \pmatrix{
C_p\zetacap_{1+\half}&C_p(\zetacap_{1+\half}+\zetacap_{2+\half})&
               \cdots&C_p(\zetacap_{L-\half}+\zetacap_{L+\half})\cr
0&C_p\zetacap_{2+\half}&
               \cdots&C_p(\zetacap_{L-\half}+\zetacap_{L+\half})\cr
\vdots& \vdots& \ddots& \vdots\cr
0&0&            \cdots&C_p\zetacap_{L+\half}\cr
}$$
 
We define the following ``explicit time-tendencies'':
 
$$\eqalignno{
\Xvec_D^n &= - \div {\vec {\v H}}_u^n - \delsq {\vec E}^n
             - \delsq \Mx[\Theta,\Phitilde] \Tvec^{n-1}
             - \gamma_u \deleight \Dvec^{n-1} - \delsq \Phi_s^n\cr
\Xvec_T^n &= - {\vec H}_{T'}^n - \Mx[D,H] \Dvec^{n-1}
             - \gamma_T \deleight \Tvec^{n-1}\cr
}$$
 
Note that the explicit $D$ tendency for each of the $L$ levels has a $(-
\delsq \Phi_s^n)$ term in it, and that is the only form in which $\Phi_s$
enters the prognostic equations. So the lower boundary condition on $\Phi$ is
equivalent to a vertically uniform explicit forcing term in $D$ tendency
equation. So it may be convenient to assume that $\Phi_s \equiv 0$, and
specify a vertically uniform explicit forcing to simulate the lower boundary
condition on $\Phi$.
 
Proceeding along similar lines as in the case with vertically integrated
divergence set to zero, we obtain the following prognostic equations for $D$
and $\Theta$:
 
$$\eqalignno{
\tt[\Dvec,n] &= {\left[1 - (\alpha 2 \Dt)^2 \gamma_u^{corr} \gamma_T^{corr}
                    \delsq \Mx[\Theta,\Phitilde] \>
                           \Mx[D,H]\right]}^{-1}\cr
             &  \quad \left( - \gamma_u^{corr} \gamma_T^{corr}
                      \delsq \Mx[\Theta,\Phitilde] (\alpha 2 \Dt) \Xvec_T^n
                      + \gamma_u^{corr} \Xvec_D^n\ \right) \cr
\tt[\Tvec,n] &= - \gamma_T^{corr} \Mx[D,H]
                    (\alpha 2 \Dt) \tt[\Dvec,n]
                + \gamma_T^{corr} \Xvec_T^n \cr
}$$
 
The treatment of $\xi$ is the same as for the case with vertically integrated
divergence set to zero.
 
 
\vfill\eject
 
\References
 
\bref Haltiner, G.J., and R.T. Williams/1980/ Numerical Prediction and Dynamic
Meteorology, {\rm second edition}. John Wiley \& Sons, 477pp.
 
\bref Holton, J.R./1979/ An Introduction to Dynamic Meteorology, {\rm second
edition}. Academic Press, 391pp.
 
\pref Lorenz, E.N./1960/ Energy and numerical weather prediction. Tellus, 12,
364-373.
 
\bref Press, W.H., B.P. Flannery, S.A. Teukolsky, and W.T. Vetterling/1986/
Numerical Recipes: the art of scientific computing. Cambridge University
Press, 818pp.
 
\pref Simmons, A.J., B.J.Hoskins, and D.M.Burridge/1978/ Stability of the
semi-implicit method of time integration. \MWR, 106, 405-412.
 
\bye
 
