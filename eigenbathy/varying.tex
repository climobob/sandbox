\section{Locally-averaged bathymetry}

  To explore the effects of averaging, it is desirable to perform 
averaging over many different scales.  On the other hand, it is
also desirable to be able to refer to an 'exact' calculation.
Averaging over many different scales argues for selecting a number
of points which is factorable many different ways.  4342 is not
(= 2*13*167), but is close to a number which is -- 4320 (= 2$^5$ *3$^3$ * 5)
Consequently, we will delete 22 points from the bathymetry -- 
3 from the Antarctic end (making minimum depth there 249 meters
instead of 130) and 19 from the Greenland end, and recompute the
full reference spectrum.

  Given our reference spectrum we can return to assessing how 
well bathymetric averaging methods perform in reproducing the
slowest eigenperiods.  (It is also always an option also to compute
degree of match on eigenmode structure, and will be done later.)
We will therefore average over 432, 288, 160, 80, 40, 20, 10, 8, 5, 4, 3, 2
points 
by each of the 4 methods, and compute the spectra for each method.
We expect to see a convergence to the exact calculation as the range 
of averaging shrinks.  The question is rather one of which method
produces the most rapid convergence.

  Figures N1-N2 show the periods of modes 0-7 for each of the 4 
averaging references, against the exact computation.  Except for
the two slowest, all show a J curve with respect to amount of averaging.
When there is relatively little averaging, the each method converges towards 
the correct period as averaging is reduced.  For all modes, averaging
1/sqrt(H) is best and remains closest to the correct value in the bowl
of the 'J' (averaging 2-160 points, 4 minutes to 5.333 degrees averaging),
with averaging 1/H being always next best, sqrt(H) next, and always worst
is averaging H.  The two slowest modes show this order of preference 
everywhere (to the over 14 degree averaging tested). 

  For extremely coarse averaging, coarser than 5-10 degrees, these
relative results are reversed for modes 2-7. 

  Figures M1-M2 show the relative error in eigenperiod for the first
100 modes (down to 30 minute period) for each averaging method, in
sequence of degree of averaging 40, 20, 10, 5, 2 points.  At 40 point
averaging we see that all averaging methods give eigenperiods which 
diverge ever-increasingly as the mode number increase above about 40
(about 75 minute period, for a nominal 4000 m depth, this corresponds
to about 900 km wavelength).  From gravest mode to about mode 25 (2 hours
period, nominal 1440 km wavelength), 1/sqrt(H) averaging is best, with
1/H averaging nearly equal.  The latter is generally best in 25-40.

  For 20 point averaging (Figure N) the point of ever-increasing divergence 
is delayed to about mode 55 (55 minutes, 660 km).  Between about 35 and
55, averaging H gives best results.  From about 20 to 25, averaging 1/H is
best.  And 0-20 is again best with averaging 1/sqrt(H).  This behavior is
repeated at 10 point averaging.  

  With 5 point averaging (10 minute resolution) 1/sqrt(H) is best for
almost all modes to about 50.  With 2 point averaging, 1/sqrt(H) is best
for almost all 100 modes.  In those cases where it isn't, 1/H averaging 
is best.

  A consistent behavior is the the 10th mode (5 hour period, nominal 
3650 km wavelength) is consistently, with respect to both averaging 
method and degree of averaging, the most difficult to obtain accurately.
Even 2 point averaging by H and sqrt(H) does not obtain this mode within
2\%.  It suffices, barely, for 1/H averaging, while 3 point averaging
(not shown) is already 2.7\% in error.  1/sqrt(H) averaging suffices
to even 10 point averaging, though next examined (20) shows 4.5\% error.



  In terms of averaging method I conclude that 1/sqrt(H) is the best
for all modes likely to be of interest for tidal synthesis or 
ocean response to meteorological forcing -- circumstances with strong
forcing at diurnal and semidiurnal period.  Taking 4 hours as the most
rapid of interest indicates that the first 13 modes are the ones of
concern.  Following [] gives the first 6.  For these slow modes, 1/sqrt(H)
is always best.  For higher modes, if the resolution of averaging is
fairly fine, the 1/sqrt(H) is again best.

