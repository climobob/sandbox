\documentclass[11pt]{article}
\usepackage{graphicx}
\usepackage{amssymb}
\usepackage{epstopdf}
\DeclareGraphicsRule{.tif}{png}{.png}{`convert #1 `dirname #1`/`basename #1 .tif`.png}

\textwidth = 6.5 in
\textheight = 9 in
\oddsidemargin = 0.0 in
\evensidemargin = 0.0 in
\topmargin = 0.0 in
\headheight = 0.0 in
\headsep = 0.0 in
\parskip = 0.2in
\parindent = 0.0in

\title{The Byte-addressable C I/O library: BACIO}
\author{Robert Grumbine, Mark Iredell, Stephen Gilbert}
\begin{document}
\maketitle

\section{Introduction}
  In the late 1990s, as the NWS central Cray computer was due to be retired, it 
became necessary to emulate the ability of the Cray for Fortran programs to 
read C-unformatted files.  The original Cray Fortran call was BAREAD, for
byte-addressable read, so a BACIO -- byte-addressable C I/O -- library was written
to preserve as transparently as possible the old codes.

  The library is in two parts, the Fortran side, which most people use, and
the C side, which manages the actual I/O.  The C side may be called directly from
Fortran programs, though it hasn't typically been used that way.


\section{Usage -- bacio.c, banio.c; baciol.c, baniol.c}

  One may directly use the C entry for the bacio library from Fortran.  
Its name, to Fortran programs, is bacio or, for structured data types, 
banio.  It is necessary at time of installation, however, to verify 
the name (banio, banio\_, ...) expected on the C side and appropriately
modify the \#define in clib.h in order to be able to link to the library.
It may be required to add a new type entirely, in which case new
calls will also need to be included below.  Only the function names themselves
need to change.  

  The arguments are, in order:\\
\begin{tabular}{lll}
INTEGER &mode & I/O Mode, see following table and sum for desired results \\
INTEGER &start&  Starting byte \\
INTEGER &size&  Size of items being read\\
INTEGER &no & Number of items requested \\
INTEGER &nactual&  Actual number read (output) \\
INTEGER &fdes & File descriptor (output) \\
CHARACTER() &fname & File name (input) \\
CHARACTER() &data & Character array for data (bacio) \\
\end{tabular}

  Note that in reading the C source code, you'll see two more 
arguments, namelen and datanamelen.  They are constructed by the 
Fortran compiler.  If you specify them, or add them to the argument 
list, you'll break the call.
 
  The following modes are not all mutually compatible.  While the 
library does attempt to check for mutually incompatible options (Readonly 
and Writeonly at the same time, for instance) you should take some care 
that you specify compatible options.  

  Different systems have different limits to how many files may be 
opened from a C program or library.  As of 21 April 2008, the NWS IBM 
SP's limit was over 20,000 files.  Redhat 7.3 was 1020 files, Mac OS-X 
10.5 limited to 252, and an SGI Origin system limited to 196 files.  
These limits are irrespective of any limits in the Fortran module, 
which currently is set to 999 files.

Table 1: Bacio and Banio operating modes

\begin{tabular}{lrl}
Modes & &\\
BAOPEN\_RONLY         &     1 & Open read-only \\
BAOPEN\_WONLY         &     2 & Open for writing only \\
BAOPEN\_RW            &     4 & Open for reading and writing \\
BACLOSE               &    8 & Close the file \\
BAREAD                &   16 & Read from the file \\
BAWRITE               &   32 & Write to the file \\
NOSEEK                &   64 & SEEK\_SET (start is byte number to seek to)\\
BAOPEN\_WONLY\_TRUNC  &    128 & Open write-only and truncate file \\
BAOPEN\_WONLY\_APPEND &    256 & Open write-only and append \\
\end{tabular}


Table 2: bacio and banio return codes

\begin{tabular}{rl}
Value & Return Code  \\
  0   & All was well                                   \\
255   & Tried to open read only \_and\_ write only     \\
254   & Tried to read and write in the same call       \\
253   & Internal failure in name processing            \\
252   & Failure in opening file   \\
251   & Tried to read on a write-only file             \\
250   & Failed in read to find the 'start' location    \\
249   & Tried to write to a read only file             \\
248   & Failed in write to find the 'start' location   \\
247   & Error in close                                 \\
246   & Read or wrote fewer data than requested        \\
102   & Massive catastrophe -- datary pointer is NULL  \\
\end{tabular}


baciol and baniol work exactly like bacio and banio, the difference
being that the arguments
start, newpos, no, nactual must now be 8 byte integers, INTEGER$*$8 in
Fortran, for instance.  Arguments mode, size, fdes are unchanged. 
This does leave a limit of 2Gb for the maximun size of things read.
Arguments for system type IBM8 are unchanged as they are already 
long long ints.


\section{Usage : baciof.f}
limit, now, of 999 files, irrespective of system.  Set in module:

      SUBROUTINE BASETO(NOPT,VOPT) \\
      SUBROUTINE BAOPEN(LU,CFN,IRET) \\
      SUBROUTINE BAOPENR(LU,CFN,IRET) \\
      SUBROUTINE BAOPENW(LU,CFN,IRET) \\
      SUBROUTINE BAOPENWT(LU,CFN,IRET) \\
      SUBROUTINE BAOPENWA(LU,CFN,IRET) \\
      SUBROUTINE BACLOSE(LU,IRET) \\
      SUBROUTINE BAREAD(LU,IB,NB,KA,A) \\
      SUBROUTINE BAWRITE(LU,IB,NB,KA,A) \\
      SUBROUTINE WRYTE(LU,NB,A) \\



\section{Conclusion}




\end{document}
