\section{Regionalization}
Region should have a name, bounding curve (including lat-long corners as case), 
area of region, area of the region that is ever covered by ice.
lat-long corners for plotting.  lat lon of type point.

Tool to quickly analyze current coverage (area, extent, fraction of domain)


\section{psuedo-intro}
A different question related to our expectations ('climate') is 'what are the
regions for sea ice'?  There are conventional definitions, such as [citeNNN].
These follow lines of latitude and longitude, and sometimes political boundaries.
The sea ice pack does not know about political boundaries, and may have little
notion of longitudes.  Latitude probably does carry some meaning to the
ice pack, as solar insolation is dependent on latitude.  I'll make some 
attempts to find objective regions.  We'll see that some correspond well to
expectations (Sea of Okhotsk, Hudson Bay stand out), some are surprisingly
large (the main Antarctic region), and some can be fairly small (a Barents Sea
region, for instance).

\section{Regionalization}


\section{Harmonic analysis of regional (inc. hemispheric) area, extent}
  \subsection{residuals + analysis}
\section{SOI-regions correlations}
\section{region-region correlations}


  Walsh climatology : Chapman, W. L. and J. E. Walsh, Long-range prediction
of regional sea ice anomalies in the arctic, Weather and forecasting, 6,
271-288, 1991.

Rogers, enso correlation

BPRC enso correlations

Gloersen, Per quasi-quadrennial oscillation papers

Mysaak et al, et al and Mysaak on long period oscillations

Icelandic long time series

SOI data cite
