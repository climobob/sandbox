\documentstyle[12pt]{article}
\textwidth=6.5in
\textheight=9.0in
\oddsidemargin=-0.00in
\topmargin=-0.25in

\begin{document}

\title{Sea ice climatology and statistical structure}
\author{Robert Grumbine}
\date{\today}
\maketitle

\pagebreak

\section{Abstract}
  Although sea ice is known to be important both in its own right (for
shipping and fishing) and as a mediator of air-sea interaction, our knowledge
of its general features, such as seasonal cycle, climatology, and statistical
structures is limited.  In some cases, there is data at hand from which
such things may be computed readily, but the computation has not been made
available in its own right.  We present such simple statistics as the
annual-average ice cover, climatological seasonal cycle, and point to point
correlations between parts of the ice pace and autocorrelations through time.
There are several reference data sets already generated, so that we consider
each of them in turn for their comparability with respect to these elementary
statistics.

  Results of this consideration are: ...

\pagebreak
\section{Introduction}
  There are an increasing number of sea ice data sets available, and 
although they have a number of points of correspondence in their production,
there are also some differences.  Some of the most elementary statistics
of the sea ice pack have not been presented in their own right; this includes
the seasonal cycle of the ice cover (although there are many works which
must have computed the cycle, it hasn't been presented outright that I've
been able to find).  In some cases, in fact, it is not clear whether 
simple statistics are even meaningful.  

\section{Simple Statistics}

  Even the simplest statistic we could consider, the annual average sea ice
concentration, raises a difficult point.  One can average the observed 
concentrations for every month of every year in a climatology, and the
result of such a computation for the northern hemisphere from the Chapman
and Walsh [1991] Arctic climatology is presented in figure 1, but this is
a distinctly unsatisfying figure.  We see that the high arctic has a high
ice concentration, as expected.  But we also see that the Bering Sea, for
example, has an XX% ice cover.  This is unsatisfactory because at no time
does the Bering sea have such a cover.  The average is reflecting an average
of N months of nearly 100% ice cover, and the rest of the year having no
ice cover.  XX% is indeed the average, but it is not something the atmosphere
or ocean ever see.  Still, if one is pressed for a single number, this
is about as good as any.  (And so on through the rest of the descendant
statistics).

  Instead of the simple average (which considers all zero % ice covers as
meaningful) we can take the average of the ice concentrations when there
is ice (i.e., to ignore the zero % ice covers in averaging) and then let
a second (required) parameter be the fraction of the time that there is
an ice cover.  The first figure is the expected ice concentration for when
there is an ice cover (a conditional ice concentration) and the second is
our expectation of there being an ice cover.  Figures 2 and 3 present the
results of such a consideration.  

  Since it is not clear {\it a priori} whether the simple mean or the conditional
mean plus expectation of cover is more meaningful, and indeed the decision
may depend on the use, I will continue with computing both sets of statistics.
Let us consider the simple mean (and descendant statistics) first, to be
followed in section B by the conditional statistics.

\subsection{Simple averaging statistics}

  In addition to the annual average, one can also compute the second moment,
and find the variance.  Figure 4 presents the variance, which ranges from zero
(land) or near zero (the high arctic) to XX (peripheral seas, particularly
Greenland).  For much of the period of this northern hemisphere climatology,
which runs from 1901 to 1990, there are no observations of the high arctic
ice cover, so that the low variability there could be an artefact.  When we
examine the data sets that do have observations there, it will turn out
that this is real (though the magnitude of variation is slightly higher)

  The southern hemisphere annual mean and variance (for the 1973-1990 
period of the southern hemisphere set) are given in figures 5 and 6.

  The annual cycle of ice cover is strong in most areas, so we also
compute the mean and variances for the monthly figures, for both hemispheres.
One important point in CW is that the figures do not represent averages 
(of any sort), rather they represent the ice cover on the last day of the
month.  This provides a point to which we will return: What is a time
period that an ice cover may be considered representative for?

  The annual cycle is presented in figures 7-54 for the NH mean (7-18),
variance (19-30), SH mean (31-42), and SH variance (43-54).  They are
also given as animations in figures 55-58, respectively.  The data for
each of the figures is available by anonymous ftp to XXX and are in GRIB
format.  Tools are available (see readme file in that directory) for
reading the data files.  The grids are different than the originals in CW,
on the grounds that this work is in support of improvements to the NCEP 
sea ice analysis and forecasting effort.  

  Aside: Since climate may change, the annual cycle above may not be the
proper one to use for the present day.  The question of climate change in
the sea ice is one to return to.  XXX.   For now we will take the null
assumption of no climate change and consider this the correct annual cycle.

  Given an annual cycle, we can remove that from our climatological series 
and consider the structure of deviations from climatology, both in time 
and in space.  We expect to find that local areas tend to be correlated 
to each other (that is, that if one point in, say, the Bering Sea is below 
normal ice cover, that the adjacent points probably are below normal as 
well).  The degree of correlation as a function of distance is a point 
of interest, however, as many analysis methods can make use of this 
information to provide improved estimates of the sea ice cover.  Also of
interest is the time correlation, as this can be used in an analysis procedure,
and it can be used as a predictor of future anomalies.  The predictive
use was the motivation in CW.

  The covariance between points is presented in figure 59 as an animation.
Each frame is the correlation between the ice cover anomaly (relative to a
point that ever gets ice cover) and every other point that ever has an ice
cover.  We see a few systematic features: 
  1) There is typically a zone around each point of a few hundred km over 
which there is substantial correlation
  2) These zones are highly anisotropic; that is, there are preferred
directions (parallel to the ice edge as a rule), along which there are 
large correlations for a greater distance than along a line perpendicular
  3) The high arctic is highly self-correlated.  This is partly required
by the data set, rather than necessarily the sea ice cover.  It is striking
nevertheless that on the order of several hundred thousand square kilometers
of ice cover can be treated essentially as a unit.    
  4) There are some very long range connections:
    a) The Baltic is anti-correlated to points on the Arctic pack near
        Svalbard and Franz Josef Land
    b) The Bering Sea shows correlation to the Labrador sea
    c) The Arctic shelf zone, both Canadian and Siberian, appears
        strongly connected, even though neither seems related to the
        central arctic or other peripheral zones. 

Point 1 is seen in the atmosphere by Thorndike [XXXX] as correlation length
scales of XXXX and XXX for pressure and temperature, respectively.


  The local time correlations are also striking.  Even at 1-3 months 
remove (figures XX to XX), there are zones for which there are large 
autocorrelations.  This was the basis of the method in CW.  Continuing 
on in time, however, we also find exceedingly high autocorrelations at 
12 months, at around 52 months (figures XX and XX, respectively).  The 
former may indicate a year to year persistence of anomalies (though we 
note that the 11 and 13 month autocorrelations are typically far less 
than at 12, as compared with 1 month lags, which give large correlations), 
a climatic trend in the seasonal cycle (if the amplitude or phase of the 
seasonal cycle were changing slowly through time, the anomalies from year 
to year would be autocorrelated), or some other feature of interest. The
latter period, 4 years and then some, might be related to long period
free changes in the ice pack, to El Nino (which averages about this period,
cite XXXX), or to other features of the climate system.  Gloersen [19XX]
detected such a period [XX months] in an analysis of XXXX.  

  We test the notion of an El Nino connection by correlating each point's
time series against the Southern Oscillation Index [cite, XXXX], for lags
0 to 72 (as is being done for the autocorrelations).  We find that indeed
there are points with fairly large (greater than 0.1) cross-correlations
between the local sea ice anomaly and the SOI.  These occur at time lead/lags
of 1.5-2.5 years (ice leading SOI, as well as lagging).  There are a number
of points, in addition, where the cross-correlation is maximum at the limits
studied (6 year ice lead over the SOI, points in the Bering/Beaufort Seas,
Siberian coast).  
Rogers [? XXXX] (1980 study) found ...
The SOI to ice relations, particularly where the ice may be leading the SOI,
is a point we'll return to.  We note, however, that the quasi-quadrennial
period is noticeably not present in the ice anomaly to SOI correlations.
Whatever it is that is driving that period in the ice cover, is apparently
not the ENSO cycle (as represented by the SOI at least).

  In the Antarctic, considering time and space correlations, the point to
point correlations (animation, figure XX) suggest:
  1) Even greater degree of anisotropy than in the Arctic.  Points many degrees
of longitude apart show high correlations while their neighbors to the
interior are highly unrelated.
  2) Ross sea and Weddell Sea ice pack (interior zone) show large regional
correlations
  3) The Weddell Sea shows the largest area of self-correlation
 
  In time, the pattern seems more involved than for the arctic.  The anomalies
show correlation at 1-2 month lags (vs. 1-3 for Arctic), but high recurring
correlations as 12, 24, 36, 48, 60, 72 months.  Unlike the Arctic, there
does not seem to be a quasi-quadrennial signal.  But also unlike the arctic,
the maxima near the annual periods seem to be rather broad, c.f., fig XX which
gives the autocorrelations for a point in the Weddell Sea.  The 4 year lag
period shows an especially high correlation, higher than the magnitude at 1,
2, or 3 years. 

  The Antarctic ice pack shows much higher cross-correlations with the SOI
than the Arcitc pack does.  The Antarctic has extensive areas of greater
than 0.3 cross correlation (magnitude).  This occurs for ice leading SOI
by about 8 months, 30 months, or lagging by about 36 months or 66-72 months.
That a connection seems to exist is not entirely surprising as [et al and
Bromwich XXXX] have seen atmospheric signals in the Antarctic being tied
to the SOI.  It is interesting, however, to note that the sea ice is typically
leading, rather than lagging, the SOI.  The atmospheric signals are seen
to lag the SOI [cite XXXX].
[Antarctic 'wave' cite: XXXX].

  
Summary:
  Some tentative conclusions from the consideration of simple sea ice
statistics:

  Interior scales of a few hundred km, and out to 3 months, for both 
hemispheres, regionally varying.

  Signs of long-range connections at several thousand km and years, in 
or between selected regions.  But we lack confidence in these results
at long range in time because of the possibility of climate changes in 
the annual cycle in the record.

  Signs of moderate correlation between SOI and Antarctic ice cover, and
weak, but nonzero correlation between SOI and the Arctic ice cover.  Some
suggestion that the Antarctic may lead SOI, and in any event, have long
range prediction based on SOI, or from SOI. This lacks confidence due to
the fact that there are only 18 years of data in the CW Antarctic data
set.

Before we make these firm conclusions we need to:
1) Consider the conditional ice cover statistics.  (If one finds the same
patterns, confidence is increased.  If the correlations are largely driven
by a single or a handful of years in which there was an anomalous ice cover,
then those correlations should be lost in the conditional ice cover fields,
as they'll only have the ice years present.)
2) Examine the series for trends in order to remove that as a 
source of long-lag correlations (as well as being an interesting 
feature in its own right)
3) Compare the statistics for the WC climatology to those from higher space
and time resolution sources.  The interior ranges of a few hundred km and
a couple of months provide bounds on how far to range.  (This is a practical
limit, as the memory and execution time requirements for 18 years of daily
data at 25 km resolution (NASA/NSIDC climatology) are much larger than for
monthly data at 60 Nm resolution, even if for 90 years (136,192 points
vs 4640, and 6570 times vs. 1080). 

  These will be done in section 2b, 3, and 4, respectively.

Aside: The high space and time correlations have implications for sea ice
prediction.  Some already noted in CW.  We will return to this point. 


\subsection{Conditional averaging stastistics}
2b.
  Figures XX to XX present the northern hemisphere's conditional average
ice concentration, conditional variance, and fraction of the time that
there is an ice cover.  We see that, as before, the high arctic is a 
zone of high concentration.  This is no surprise, as the high arctic is
perennially ice covered.  It is in looking to the peripheral seas that
we find the interesting differences with respect to the simple averages.
Hudson Bay, Bering Sea, Sea of Okhotsk, and Baltic Sea are now all zones 
of high conditional ice concentration.  That is, if there is an ice cover
in these areas, the concentration is high.  We see from the coverage figures
that the frequency of having an ice cover varies substantially between 
thes areas, as expected.  

  Most interesting are the zones with low average
conditional ice cover.  This includes in the Baffin Bay/Davis Strait, and 
the Greenland, Iceland, Norwegian, and Barents seas.  It is common for
there to be an ice cover in these areas (figure XX) but the concentration
varies substantially even when there is a cover.   

  The Antarctic shows similar behaviors.  The zones of highest conditional
ice cover are mostly towards the periphery (where it is somewhat unusual 
for there to be an ice cover, but where there is one, it is substantial) or
in the perennially covered zones.  Between the two is an extensive zone 
of moderate conditional concentration.  The conditional variance shows,
again, low variance in the high average conditional concentration areas,
and highest variance towards the periphery.  

Annual cycle:
  So far there are no surprises, given the differences we expected to see,
between the conditional and unconditional ice concentrations.  There
is some sign of at least an aesthetic reason to prefer the conditional
concentrations in the peripheral zones.

  The seasonal cycle is given in figures XX-XX for the northern hemisphere
mean, variance, and probability of coverage.  Figures XX to XX provide this
information for the southern hemisphere.  The data for this annual cycle 
are in grib format in XX/XXX.

Deviations from climatological cycle:

  Note only computed for series greater than XX% complete ...

  Again we removed the average ice concentration (this time the conditional
concentration) for each month from the time series and looked at the 
temporal and spatial correlations between the deviations from climatology.
The covariance between points for the northern and southern hemispheres is
given in figures XX and XX as animations.  We see that some of the features
remain from before:
  1) There is still a zone of a few hundred km over which there is substantial
correlation.  The zone and correlations are typically smaller than for 
the unconditional averages, however, indicating that a significant portion 
of the correlations seen previously were due to regions being ice free 
at the same time.   
  2) The anisotropy of the correlation fields is even greater than for
the unconditional averages.  Again, the preferred directions tend to 
parallel the ice edge. 
  3) The high arctic remains highly self correlated.
  4) Some long range correlations remain, but most are weaker or removed.
No new zones of correlation appear.
    a) The Baltic sea loses its correlation to other areas.
    b) The Bering Sea shows no long range correlations
    c) The Arctic shelf zone, both Canadian and Siberian, still appears
        strongly connected, even though neither seems related to the
        central arctic or other peripheral zones.

We also find that the 1-3 month autocorrelations remain large in the arctic.
Again, the arctic shows signs of returning correlation at multiples of
12 months, and in the vicinity of 52 months.  In repeating the correlations
with respect to the SOI, we find substantially different patterns of
correlation.  The mangitude remains about the same, correlations peaking
at about +- 0.1, but there is now no artefact of peaking towards the end
of the lag list (+= 72 months).  The lag relationship indicates ice leading
SOI, with maxima in the range of 8-20 months, somewhat shorter lead than
was indicated in the unconditional averages.  Areas with particularly high
relation to the SOI are to the NE of Greenland, and in the Beaufort gyre.
(Figure XXX).  The changes in SOI correlation pattern is our second significant
difference between using conditional and unconditional statistics.

  Considering the Antarctic space correlations, we find that our
conclusions from section 2a remain:
  1) Even greater degree of anisotropy than in the Arctic.  Points many degrees
of longitude apart show high correlations while their neighbors to the
interior are highly unrelated.
  2) Ross sea and Weddell Sea ice pack (interior zone) show large regional
correlations
  3) The Weddell Sea shows the largest area of self-correlation

with the modifier, as for the arctic, that the range and magnitude of 
correlation is reduced in the conditional statistics.

  The Antarctic time autocorrelations for conditional ice cover show
even more rapid decline than for the unconditional, with 1-2 month lag
being the limit.  The strucures at annual returns are less apparent than
for the unconditional, but it does remain the case that annual return times
show large correlations compared to other periods.  The 52 month period still
does not appear in the Antarctic record (though with a record only 4 times
that long, and crude statistics, one is hesitant to draw conclusions).

  Correlation to SOI still shows magnitudes greater than 0.3, and maxima
at about 7-8 months and 30 months ice lead, and for 36 months SOI leading
the ice.  The 66-72 month maximum goes away.

\subsection{Ice/No Ice as a meaningful map}
  Section 2b, however, drew its conclusions with respect to the conditionally-
averaged ice concentrations.  It seems clear, from at least some of the 
differences between that and the unconditional averages, that the ice/no 
ice status is both interesting and may contain much of the information 
present in the observational record.  We therefore examine the binary set
itself.   

  Since binary data are different than the (in principle) continuous ice
concentration data, we will pause to consider what differences in analysis
we will need.  First, any point which always has an ice cover contains no
information.  This contrasts with the concentration case, were even perennially
ice-covered points are informative as the concentration may change.  Or, if
the concentration didn't change, this too would be physically (if not 
statistically) interesting.  Second, binary series lend themselves to an
interpretation as a probability of ice cover.  This is reminiscent of
precipitation prediction, where we are given a probability of rain and
an indication of how much rain there would be if it did rain.  We'll return
to this point in model verification and testing as precipitation forecasting
has a long verification literature.  Third, we need to find a method
for examining point to point coherency/correlation, as the usual definition
is for continuous variables.  And fourth, we need methods of comparing the
binary series to a continuous one (as in the SOI).  Each of these is a 
manageable point, but they deserve consideration as they're not just like
the continuous case.

  We'll consider points 2-4 in reverse order, given point 1.  The relation
to a continuous series, SOI being our type case, is a halfway step.  One
of the series is still continuous, as before.   
XXXXXXXXXXXXXX
  Before proceeding, there is a further step which we must consider.
That is, seasonality makes a tremendous difference for all our binary
stastics.  In the case of unconditional ice concentrations, we consider
all the zero ice concentrations to be meaningful, and so no special care
is mandatory.  In the conditional ice concentrations, we only worry about
what the ice does {\it when it is present}, so again, the figures are
well-defined.  For the binary series, we must deal with the fact that 
it is very likely that the one time a given point has ice will be in
mid-winter.  Conversely, the one (or few) times a point is ice free
will be in mid-summer.  Consequently, any effect which does have a seasonal
bias (as El Nino, named for a typical December maximum XXX) must be
broken out by month of the year.


  Point 3 can be approached by considering independance, which is
well-defined for discrete variables.  If the two variables are
independant (which we'll take as our proxy for uncorrelated), then
the probability that both X and Y occur (point X has ice at the same
time as point Y) is simply the product of the probability that
X has ice (in general) times the probability that Y has ice (in general).
If we consider the process as a Bernoulli procedure, we can also 
approximate the distribution of the differences between P(XY) and
P(X)*P(Y) as being gaussian, and hence arrive at a z-like statistic:
z = (P(XY) - P(X)*P(Y))/sqrt(N*p*(1-p)), where p will need consideration,
but which we will take to be P(XY) for now.  In all cases, however, 
we will be working with the sample estimate of the probabilities.




\section{Why is there long, integer-year autocorrelation?}
sect 3.

\section{Statistics from higher resolution sea ice information}
sect 4.
-- trend (annual cycle) removal per previous section
-- recompute correlation ranges, in space and time.
Regions:
Eastern North America:
  Great Lakes
  Gulf of St. Lawrence
  Labrador Sea/Coast
  Hudson Bay
  Baffin Bay
  
Scandanavian Seas:
  Greenland Sea
  Iceland Sea
  Norwegian Sea
  Barents Sea
  Baltic Sea

Pacific:
  Yellow Sea
  Sea of Japan
  Okhotsk Sea
  Bering Sea
  
Eurasia:
  White Sea
  Kara Sea
  Laptev Sea
  Chukchi

N. NA
  Beaufort
  Canadian Archipelago

High Arctic

Asian Seas:
  Aral Sea
  Caspian Sea
  Sea of Azov

  
\section{Re-examining the data quality}
sect 5.

  In the previous sections, we focussed on developing reasonable statistical
method for dealing with sea ice concentration information.  A secondary
focus was to obtain some reasonable order of magnitude and etiology regarding
the ice pack statistics.  The conclusions regarding physical meaning of the
relations, however, have had to be tentative because we did not examine in
any detail the quality of the data on which the methods were being applied.

  Now that we have some notions of what we are looking at -- a data set
for which the statistics are anisotropic, inhomogeneous, and non-stationary,
with length scales of dozens to a few hundred km, time scales of weeks, 
and which can be reasonably described in terms of probability of ice cover
with conditional concentrations -- we can proceed.  We see, for example, that
it is important to establish the degree to which data source forced long
distance or long-time correlations in the CW data set.

....
After limiting ourselves to data sources XXX in the CW set, we find XXX.

  

\section{Conclusions}

\subsection{Initial conclusions}

\subsection{Further research}
Ice field prediction null forecaster
SOI null forecaster
Analysis method incorporating space-time correlations and conditional coverage
Reconstruction from ice edge (conditional coverage)


  

Data sets:
  Walsh climatology : Chapman, W. L. and J. E. Walsh, Long-range prediction
of regional sea ice anomalies in the arctic, Weather and forecasting, 6,
271-288, 1991.
 
  NASA on the fly climatology

  NASA revised climatology

  NCEP analysis

  USGS monthly
 

 

\pagebreak
Bibliography

 Chapman, W. L. and J. E. Walsh, Long-range prediction
of regional sea ice anomalies in the arctic, Weather and forecasting, 6,
271-288, 1991.

Rogers, enso correlation

BPRC enso correlations

Gloersen, Per quasi-quadrennial oscillation papers

Mysaak et al, et al and Mysaak on long period oscillations

Icelandic long time series

SOI data cite


\end{document}
